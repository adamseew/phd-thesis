
%%%%%%%%%%%%%%
%            %
% Guidance   %
%            %
\chapter{\color{red}Implementation}
\label{app:imp}

%\lettrine{A}{a}


\section{Energy Models}

To build the model in \matlab one can use the function \stt{build\_model} from \fref{lst:build_model}{Listing}.

\begin{lstlisting}[language=Matlab,caption={[Function to create the energy model]Function \stt{build\_model} that creates the energy model.},captionpos=b,label=lst:build_model]
function [model] = build_model(dat)
    
    m = 2*dat.r+1;
    
    Aj = @(dat.omega,j) [0 dat.omega*j;-dat,omega*j 0];
    A  = zeros(m);
    A(1,1) = 0;

    B = [zeros(1,dat.rho) ones(1,dat.sigma)];
        
    C = [1];
    
    for i = 1:dat.r
        A(2*i:2*i+1,2*i:2*i+1) = Aj(dat.omega,i); 
        C = [C 1 0];
        B = [B;zeros(1,dat.rho+dat.sigma)];
    end

    model.A = A;
    model.B = B;
    model.C = C;

    model.est_u = @(c) dat.nu*c+dat.tau;
    model.u = @(u1,u0) u1-u0;

end
\end{lstlisting}

The function builds the matrices from \frefeq{eq:state-perf}. In particular matrix $A$ from \frefeq{eq:mat_A}, $B$ from \frefeq{eq:mat_B}, and $C$ from \frefeq{eq:mat_C}. The term $1/T$ is missing in \stt{C} as we motivated in the proof of \fref{lem:eqv}{Lemma} in \frefeq{eq:mat_C_generic}. The nominal control from \frefeq{eq:state-control} is \stt{u}, and the energy estimate of a given control sequence at time instant $t$ and the previous time instant $t-1$ from \frefeq{eq:estimate-control} is \stt{est\_u}. 

The function requires the structure \stt{dat} illustrated in \fref{lst:struct_dat}{Listing}. The structure contains some information about the model. In particular, \stt{r} is the order $r$ of the model from \fref{sec:periodic-model}{Subsection}, \stt{omega} is the angular frequency $\omega$, \stt{rho} the number of path parameters to alter the path $\rho$, and \stt{sigma} the number of computation parameters to alter the computations $\sigma$. The path parameters are defined in \frefeq{eq:path-params}, and the computation parameters are defined in \frefeq{eq:comp-params}. Furthermore, \stt{delta\_T} is the sampling step $\Delta t$, and \stt{nu} and \stt{tau} are scaling factors from \frefeq{eq:scaling}.

\begin{lstlisting}[language=Matlab,caption={[Struct used to build the model]Struct \stt{dat} used by function \stt{build\_model} to build the model.},captionpos=b,label=lst:struct_dat]
dat.r = 3;
dat.omega = 2*pi/period;
dat.rho = 1;
dat.sigma = 1;
dat.delta_T = .01;
dat.nu = nu;
dat.tau = tau;
\end{lstlisting}

The function \stt{build\_model} and the structure \stt{dat} can be used along an initial guess of parameters and a time horizon to model the future energy consumption using the Euler or Runge-Kutta methods for numerical integration. The Euler method is described in \fref{sec:euler}{Subsection} and implemented in \fref{lst:euler}{Listing}, the Runge-Kutta method in \fref{sec:rk4}{Subsection} and \fref{lst:rk4}{Listing}. For the Runge-Kutta method, we use the fourth-order fixed size standard algorithm. 

\begin{lstlisting}[language=Matlab,caption={[Euler method for numerical integration]Euler method for numerical integration of the model.},captionpos=b,label=lst:euler]
function [q,y] = evolve_model_euler(model,dat,init)

    m = 2*dat.r+1;
    Ad = model.A*dat.delta_T+eye();

    q0 = init.q0;

    q = q0;
    y = C*q0;
    
    est_u0 = init.est_u0;
    est_u1 = model.est_u(init.c_chain(1));

    i = 2;
    
    for init.t0+dat.delta_T:dat.delta_T:init.tf
        q0 = Ad*q0+model.B*model.u(est_u1,est_u0);
        est_u0 = est_u1;
        
        if i <= size(init.c_chain,2)
            est_u1 = model.est_u(init.c_chain(i));
            i = i+1;
        end

        q = [q;q0.'];
        y = [y;C*q0];
    end
end
\end{lstlisting}

The function \stt{evolve\_model\_euler} in \fref{lst:euler}{Listing} evolve the model in \frefeq{eq:state-perf} from an initial time $t_0$ to the final time $t_f$. This information is passed to function using the structure \stt{init} from \fref{lst:init}{Listing}.

\begin{lstlisting}[language=Matlab,caption={[Structure \stt{init} with numerical integrator's initializations]Structure \stt{init} with numerical integrator's initializations.},captionpos=b,label=lst:init]
init.t0 = 0;
init.tf = 60;
init.q0 = q0;
init.est_u0 = est_u0;
init.c_chain = c_chain;
\end{lstlisting}

In \fref{lst:init}{Listing}, the field \stt{t0} indicates the initial time $t_0$ (the mathematical simulation starts at $t_0+\Delta t$ assuming the instant $t_0$ corresponds to the state $\mathbf{q}(t_0)$). The field \stt{tf} indicates the final time $t_f$ when the simulation stops. Both fields are expressed in seconds. The field \stt{q0} indicates the initial state guess at time instant $t_0$, $\mathbf{q}(t_0)$, and finally \stt{c\_chain} the sequence of controls. If we assume that the simulation horizon is $N:=t_f-t_0$ with $(t_f-t_0)\in\mathbb{R}_{>0}$, then the user is expected to provide $N-1$ controls. Finally, the field \stt{est\_u0} contains the control energy estimate at time $t_0$, the value $\hat{\mathbf{u}}(t_{-1})$. If $t_0$ is the initial time step, then \stt{est\_u0} is set to zero.  

\begin{lstlisting}[language=Matlab,caption={Runge-Kutta fourth order method for numerical integration of the model.},captionpos=b,label=lst:rk4]
% TODO
\end{lstlisting}

Before calling structure \stt{dat}, it is necessary to define the scaling factors $\nu_1,\nu_2,\dots,\nu_{\rho}$ and $\tau_1,\tau_2,\dots,\tau_{\rho}$ for the path parameters, and $\nu_{\rho+1},\nu_{\rho+2},\dots,\nu_{\rho+\sigma}$ and $\tau_{\rho+1},\tau_{\rho+2},\dots,\tau_{\rho+\sigma}$ for the computation parameters. If we use \fref{lst:init}{Listing} there is one path and one computation parameters. 

\begin{lstlisting}[language=Matlab,caption={[Implementation of path and computation parameters scaling factors]Implementation of path and computation parameters scaling factors.},captionpos=b,label=lst:nu_tau,escapechar=|]
nu1 = -(max_t-min_t)/min_c1;|\label{inlst:path-params}|
tau1 = -min_c1*(min_t-max_t)/min_c1+min_t;

nu2 = (g_max_c2-g_min_c2)/(max_c2-min_c2);|\label{inlst:comp-params}|
tau2 = min_c2*(g_min_c2-g_max_c2)/(max_c2-min_c2)+g_min_c2;

nu = [nu1;nu2];
tau = [tau1;tau2];
\end{lstlisting}

The scaling factors thus can be defined according to \frefeq{eq:scale-traj} for the path parameter on \fref{inlst:path-params}{Line} and according to \frefeq{eq:scale-comp} for the computation parameter on \fref{inlst:comp-params}{Line}.

Let us assume there is one path parameter $c_1$ which doesn't change for the stages $i=\{1,2,\dots,l\}$. \stt{max\_t} and \stt{min\_t} are the maximum and minimum times that correspond to the time needed to hypothetically execute the path with parameter $c_1=\overline{c}_1$ and with parameter value $c_1=\underline{c}_1$. A guess for these values can be obtained empirically; we obtained them by running the simulator. They correspond to $\overline{t}$ and $\underline{t}$ in \frefeq{eq:scale-traj}.

Let us further assume there is one computation parameter $c_2$ which doesn't change for the stages $i=\{1,2,\dots,l\}$. \stt{max\_c2} and \stt{min\_c2} are the minimum and maximum configuration of the computation parameter $c_2$ defined in \fref{def:stage}{Definition}. \stt{g\_max\_c2} and \stt{g\_min\_c2} are values retrieved from \powprof{} and they correspond to $g(\overline{c}_2)$ and $g(\underline{c}_2)$ in \frefeq{eq:scale-comp}. Function $g$ is formally defined in \fref{def:comp-ener}{Definition}.

A possible call to the functions \stt{build\_model} and \stt{evolve\_model\_euler} or \stt{evolve\_model\_rk4} from \fref{lst:build_model}{Listing} and \fref{lst:euler}{Listing} or \fref{lst:rk4}{Listing} is illustrated in \fref{lst:call_model}{Listing}.

\begin{lstlisting}[language=Matlab,caption={[Example to build and evolve the model]An example to build a differential model and evolve it over a horizon $N$.},captionpos=b,label=lst:call_model]
model = build_model(dat);
[q,y] = evolve_model_euler(model,dat,init);
\end{lstlisting}

A generic routine to plot the resulting modeled energy and the coefficients $\mathbf{q}$ is illustrated in \fref{lst:model_plot}{Listing}.

\begin{lstlisting}[language=Matlab,caption={[Plot of the modeled energy and coefficients]A generic polotting routine for them modeled energy and coefficients evolution.},captionpos=b,label=lst:model_plot]
time = dat.t0:model.delta_T:dat.tf;

figure;
plot(time,y)
title('energy evolution');
ylabel('power (W)');
xlabel('time (sec)');

figure;
m = 2*dat.r+1;
t = tiledlayout(m,1);
title_str = {'alpha','beta'};
nexttile,plot(time,q(1,1:end))
title(strcat(title_str(1),0));

for i=2:m
    nexttile,plot(time,q(i,1:end))
    title(strcat(title_str(mod(i,2)+1),i-1));
end

title(t,'coefs evolution');
xlabel(t,'time (sec)');
ylabel(t,'value');
\end{lstlisting}

We describe the estimation of $T$, \stt{period} in structure \stt{dat}, in \fref{sec:imp_period_est}{Subsection}. We generate the $N-1$ controls in \fref{sec:imp_opt}{Section}. For benchmarking, one can define the lowest (or similarly the highest) level of parameters with \fref{lst:lowest_control}{Listing}. It has to be executed before defining the structure \stt{init}.

\begin{lstlisting}[language=Matlab,caption={[Lowest control chain]Implementation of the lowest possible control for benchmarking.},captionpos=b,label=lst:lowest_control]
c_chain = min_c1*ones(1,N-1);
c_chain = [c_chain;min_c2*ones(1,N-1)];
\end{lstlisting}

\section{\color{red}State Estimation}

\subsection{\color{cyan}Estimation of the period}
\label{sec:imp_period_est}

\begin{lstlisting}[language=Matlab,caption={[Estimation of the period]Estimation of the period.},captionpos=b,label=lst:period_est]
% iterating trajectories in the plan to get the constant n (to measure the period)    
d = [];
p = [0; 0];
n = 0;
    
for traj = transpose(path) % per each trajectory
            
    traj = split(traj,';');
    traj = str2sym(traj(3));
        
    x = p(1);
    y = p(1);
    di = double(subs(traj));
    x = p(1)-sp(1);
    y = p(2)-sp(2);
        
    if ismember(double(subs(traj)),d)
        break;
    else
        d = [d di];
    end
        
    n = n+1;
end
    
fprintf('n is %d\n', n);
\end{lstlisting}

\subsection{\color{cyan}Estimation of the state}

\begin{lstlisting}[language=Matlab,caption={[Estimation of the state using Kalman filter]Estimation of the state using Kalman filter.},captionpos=b,label=lst:state_est_kf]
minus_q1 = Ad*q0+B*est_u();

minus_P1 = Ad*P0*Ad.'+Q;

K1 = (minus_P1*C.')*(C*minus_P1*C.'+R)^-1;
q1 = minus_q1+K1*(pow(end)-C*minus_q1);
P1 = (eye(size(q0,1))+K1*C)*minus_P1;

q0 = q1;        
P0 = P1;
            
y = [y;C*q1];    
q = [q q0];
\end{lstlisting}


\section{\color{cyan}Guidance}


\begin{lstlisting}[language=Matlab,caption={[Guidance vector field]Guidance vector field.},captionpos=b,label=lst:gvf_fun]
function [u_theta] = gvf_control_2D(p,dot_p,ke,kd,path,grad,hess,dir)
    
    if (nargin(path) > 1) % function handle has two arguments (circle)
        e = path(p(:,1),p(:,2));    
        n = grad(p(:,1),p(:,2));
    else % one argument (line)
        e = path(p(:,1));    
        n = grad();
    end
        
    H = hess();
    E = [0 -1;1 0];
    tau = dir*E*n;
    
    dot_pd = tau-ke*e*n; % (7)
    ddot_pd = (E-ke*e*eye(2))*H*dot_p-ke*n'*dot_p*n; % (10)
    ddot_pdhat = -E*(dot_pd*dot_pd')*E*ddot_pd/norm(dot_pd)^3; % (9)
    
    dot_Xid = ddot_pdhat'*E*dot_pd/norm(dot_pd); % (13)
    
    u_theta = dot_Xid+kd*dot_p'*E*dot_pd/(norm(dot_p)*norm(dot_pd)); % (16)
end
\end{lstlisting}

\begin{lstlisting}[language=Matlab,caption={[Call to the guidance function and dynamics evolution]Call to the guidance function and dynamics evolution.},captionpos=b,label=lst:gvf_call]
u_theta = gvf_control_2D(log_p(:,end).',pdot,ke,kd,path,grad,hess,dir);
        
th_delta = kp*(hd-h)-kvv*vv;
wbx = dot(w,[cos(theta),sin(theta)]);
vs = cth*(th_nominal+th_delta)+wbx;
L = cl*vs*vs;
av = 1/m*(L-W);

vv = vv+av*delta_T;
h = h+vv*delta_T;

% Horizontal unicycle model
sh = cth*(th_nominal+th_delta);
pdot = sh*[cos(theta);sin(theta)]+w;

p = p+pdot*delta_T;
theta = theta+u_theta*delta_T;  
theta = wrapToPi(theta); % normalizing between -pi and pi
\end{lstlisting}

\section{\color{cyan}Optimal Control Generation}
\label{sec:imp_opt}

\begin{lstlisting}[language=Matlab,caption={[Model predictive control]Model predictive control.},captionpos=b,label=lst:mpc]
function [c_chain q_chain] = mpc(min_c1,max_c1,min_c2,max_c2,c1,c2,N,eu,b0,b,qc,int_v,q0,Ad,B,C,u,est_u,k,delta_T,r)
    import casadi.* % import casadi for optimal control

    interval = k+delta_T:delta_T:k+N; % no k+1 in the formula as we add the 
    % initial condition implicitly
    hh = length(interval);

    opti = casadi.Opti(); % define opt problem
    Q = opti.variable(2*r+1,N);
    U = opti.variable(2,N-1);
    L = opti.variable(1,N);
    log_Q = opti.variable(2*r+1,hh);
    log_b = opti.variable(1,hh);

    opti.set_initial(Q(:,1),q0);

    opti.subject_to(C*Q(:,1)-b0*qc*int_v <= 0);
    opti.subject_to(U(1,:) == c1);
    opti.subject_to(min_c2 <= U(2,:) <= max_c2);
    opti.subject_to(min_c1 <= U(1,:) <= max_c1);

    eeu = eu; % control estimate (from model)
    eeeu =  est_u(c1,c2);

    jjj = 1;
    dQ = q0; % initial state
    log_Q(:,1) = dQ; 
    
    for jj=1:hh-1
        dQ = Ad*dQ+B*u(eeeu,eeu);
        eeu = eeeu;
        b0 = b0+delta_T*b(C*dQ); % battery dynamics

        log_Q(:,jj+1) = dQ;
        log_b(jj+1) = b0; 

        if mod(jj,1/delta_T) == 0 % every sum in the MPC
            L(jjj) = C*dQ;

            if (jjj < N)                
                eeeu = est_u(U(1,jjj),U(2,jjj));
            end

            opti.subject_to(Q(:,jjj+1)-dQ == 0); % dynamics const
            opti.subject_to(C*Q(:,jjj+1)-b0*qc*int_v <= 0); % battery const

            jjj = jjj+1;
        end             
    end

    opti.minimize(-sum(L.^2));
    opti.solver('ipopt');

    try
        sol = opti.solve();
        c_chain = sol.value(U); % optimal control u on N
    catch
        c_chain = [ones(1,N-1)*min_c1;ones(1,N-1)*min_c2]; % there is no control which sattisfies consts
    end            

    q_chain = opti.debug.value(Q);
end
\end{lstlisting}
