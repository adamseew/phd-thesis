
%
%%%%%%%%%%%%%%%%%%%%%%%%%%%%%%%%%%%
%                                 % 
% Summary and Future Directions   %
%                                 %
%%%%%%%%%%%%%%%%%%%%%%%%%%%%%%%%%%%
%
% Thorough proofreading: 26/10/2021 19:05 
% Integrated UPS comments
% ----------------------------------
% Quality check: 26/10/2021 22:19 PASS
%
\chapter{Summary and Future Directions}
\label{cp:conc}

\begin{chapquote}{\cite{ondruska2015scheduled}}
  ``The energy budget for sensing and computing is commensurate with  that of actuation--such as is typically the case for planetary rover missions.''
\end{chapquote}
  
\vspace*{1em}


%Answer the research questions
%Show how you have addressed your aims and objectives
%Explain the significance and implications of your findings
%Explain the contribution the study makes
%Explain the limitations of the study
%Lay out questions for further research

\lettrine{P}{lanning-scheduling} energy awareness demonstrates improved robustness to in-flight failures in the autonomous use case, mitigating the adverse effect of environmental interferences on battery life and overall flight span. In the previous chapters, we progressively built and showcased our approach using the use case both in simulation and some early flight of the Opterra fixed-wing autonomous aerial robot. The approach varies both the path and the schedule while exploiting differential periodic, computations, and battery models. In this chapter, we summarize our work, discuss the outcomes, and propose future directions and conclusions in \frefm{sec:conc-summ}{sec:conc-conc}{Sections}.


%%%%%%%%%%%%%%%%%
\section{Summary}
\label{sec:conc-summ}

The approach in this work provides an energy-aware path and a power-saving schedule altogether. Past studies often derive one of these aspects, whereas the analysis of the interactions of the two is less common~\citep{brateman2006energy,sudhakar2020balancing}. To this end, our work focuses on coverage path planning (\Gls{acr:cpp})\findex{coverage path planning}--a common problem in the planning literature where it is required to visit each point in a given space~\citep{choset2001coverage,galceran2013survey}--in a precision agriculture use case. Here, a fixed-wing aerial robot covers a given agricultural field, detects ground hazards, and communicates with other ground-based actors. Although use-case specific, the approach is generic in terms of computations and battery modeling as well as CPP. The guidance and differential periodic energy modeling can be applied to a broader class of energy-constrained autonomous mobile robotics use cases.

To derive energy-aware coverage planning and scheduling for autonomous aerial robots, we defined some basic constructs in \fref{cp:pb}{Chapter}. These included the concepts of computations and motion along with their energies in \fref{sec:definitions}{Section}. Indeed their close interaction is the basis of this work. Computations are energy-expensive computational tasks that we aim to schedule energy-wise along with the motion on a coverage path. The coverage path is a succession of continuous and twice differentiable path functions in \fref{sec:path-functions}{Section}. These functions are wrapped with computations in stages in \fref{sec:defs-stages-triggs}{Section}, along with parameters for scheduling and coverage replanning. The succession of stages forms a plan in \fref{sec:plan}{Section}, and the progression from one stage to another happens in the proximity of triggering points. The plan can be expressed as a mere succession of stages or using primitive paths with a shift. In \fref{sec:2pbs}{Section}, we formally defined the problems of coverage planning and energy-aware coverage replanning and scheduling.

We discussed the state of the art spanning broadly in \fref{cp:soa}{Chapter}. In \frefm{sec:soa-ene-mod}{sec:soa-ene-bat}{Sections}, we reviewed the literature for computations energy and battery modeling. We focused on energy models for heterogeneous elements, involving CPUs and GPUs, and on abstract battery models. We then examined motion planning for mobile and aerial robots in \frefm{sec:soa-motion-pl}{sec:soa-aerial-pl}{Sections}, including planning-scheduling interactions in the robotics literature in \fref{sec:soa-comp-motion-pl}{Section}.

In \fref{cp:model}{Chapter}, we provided energy models for future energy estimations of computations, motion, and battery before digging into the technicalities of coverage planning and energy-aware replanning. The computations energy model in \fref{sec:comp-ener-model}{Section} models heterogeneous elements (i.e., CPUs and GPU) of the computing hardware and provides overall energy, average power, and battery state of charge (\Gls{acr:soc}) in the function of a software configuration. It utilizes a two-layer architecture, where the top layer incorporates multiple bottom layers. This approach is implemented in an automatic modeling and profiling tool named \powprof~\citep{seewald2019coarse,powprofiler}. We then derived a battery model in \fref{sec:battery-model}{Section} from an abstract model in the literature termed ``Rint''. We proposed a differential periodic energy model for the motion in \fref{sec:mot-ener-model}{Section}, exploiting energy characteristics of the coverage from some empirical observations. The motion energy model includes the computations energy via the control, modeling the energy effect of different schedules.

We discussed the coverage planning and scheduling, energy-aware replanning, and the guidance on a coverage path in \fref{cp:dyn}{Chapter}, providing the solution to the original problems in \fref{cp:pb}{Chapter}. First, in \fref{sec:gvf}{Section}, we discussed how to guide the aerial robot given a set of paths in the plan using the theory of vector fields. We derived a coverage plan that visits all the points in a given space in \fref{sec:cov-path-plan}{Section}, using cellular decomposition and a modified coverage motion for fixed-wing aerial robots with variable coverage that we termed Zamboni-like motion. We included all the past constructs in \fref{sec:mpc}{Section} for energy-aware coverage replanning and scheduling. Here, we replanned the original path using the computations and motion energy along with the battery models. We used a modern optimal control and state estimation technique robust to uncertainty. 

Finally, in \fref{cp:res}{Chapter}, we discussed the experimental setup and results. In \fref{sec:res-ene-comps}{Section}, we provided the computations energy models derived with the \powprof{} tool. We saw case studies of motion and periodic energy models in \fref{sec:res-ene-mot}{Section}, and finally, discussed the results for energy-aware coverage planning and scheduling in \fref{sec:res-dyn}{Section}.


%%%%%%%%%%%%%%%%%%
\section{Outcomes}

This works contributes to the existing robotics literature on planning-scheduling energy awareness~\citep{mei2005case,mei2006deployment,brateman2006energy,zhang2007low,sadrpour2013experimental,sadrpour2013mission,ondruska2015scheduled,lahijanian2018resource,sudhakar2020balancing} by proposing an approach for energy-aware coverage planning for autonomous aerial robots.

Specific contributions in order as they appear in the previous chapters include:
\begin{enumerate*}[label={(\alph*)},font={\textit}]
  \item the introduction of the concept of replanning a variable CPP along with scheduling~\citep{seewald202Xenergy},
  \item the formal definition of no-interest zones (NIZs) -aware schedules for further energy savings by selectively running computations only when necessary~\citep{seewald202Xenergy},
  \item the design and implementation of an automatic modeling and profiling tool named \powprof{}~\citep{seewald2019coarse} for various computing hardwares~\citep{powprofiler} that derives power, energy, and battery SoC as a function of software configurations,
  \item the extension of the \powprof{} tool to dataflow computational networks~\citep{seewald2019component} and to the Robot Operating System (\Gls{acr:ros}) middleware~\citep{zamanakos2020energy},
  \item the adaptation of the ``Rint'' equivalent electrical circuit in the literature into the \powprof{} tool~\citep{seewald2019coarse} and in the coverage replanning problem~\citep{seewald202Xenergy},
  \item the derivation of a differential periodic energy model~\citep{seewald202Xenergy} based on empirical observations~\citep{seewald2020mechanical} for future energy predictions and energy predictions under variable schedules for autonomous coverage, potentially applying to a broader class of mobile robotics use cases~\citep{seewald2020beyond},
  \item the adaptation~\citep{seewald202Xenergy} of an existing approach for guidance~\citep{de2017guidance} based on vector fields to coverage plans composed of multiple paths,
  \item the introduction of the Zamboni-like motion\findex{Zamboni-like motion} for variable coverage~\citep{seewald202Xenergy} for aerial robots with strict turning radius constraints, and
  \item {\itshape the derivation of an algorithm for replanning of the coverage and scheduling inflight and under strict energy constraints}~\citep{seewald202Xenergy}.
\end{enumerate*} 


%%%%%%%%%%%%%%%%%%%%%%%%%%%
\section{Future Directions}
\label{sec:conc-future}

Multiple observations and open questions regarding possible future studies were made in the previous chapters. These include further developments spanning from coverage planning to broad planning-scheduling considerations.

% non aerial mobile robots OK

We hypothesized a possible generalization to other mobile robots. Indeed CPP in the literature applies to diverse use cases. These include decontamination, mine clearance, oceanographic mapping, cleanup, and others~\citep{choset1998coverage} and involve various mobile robots~\citep{galceran2013survey}. Although the Zamboni-like motion in \fref{sec:cov-motion}{Section} is best suited for aerial robots with strict turning radius constraints, the concept of variable coverage can be applied broadly. For instance, the boustrophedon-like motion for variable coverage in \fref{sec:path-wise}{Section}, to rotary-wings aerial robots and other mobile robots.

In our past brief work~\citep{seewald2020beyond}, we discussed the approach's applicability to the planetary exploration context. An unexplored terrain can be covered with variable coverage by trading a greedy technique with more complex planning. Schedules can vary by, e.g., pattern detections of features of interest or perception~\citep{ondruska2015scheduled}. Recent contributions are increasingly proposing onboard autonomy with artificial neural networks~\citep{gankidi2017fpga} and other machine learning techniques~\citep{ono2020maars} for future space exploration missions such as Sample Retrieval and Lander mission~\citep{muirhead2019mars}. In the latter, the robot is to be solar-powered~\citep{higa2019vision}. Planning-scheduling energy awareness might pose an advantage for these systems, despite past instances carrying limited onboard resources due to radiation levels and temperature changes~\citep{bajracharya2008autonomy,gankidi2017fpga}. Furthermore, the average motion energy reported in a recent study~\citep{ishigami2011path} decreased considerably compared to the past literature~\citep{krotkov1992performance}.

% different energy models OK

The differential periodic energy model in \fref{sec:mot-ener-model}{Section} is generic, modeling systems energy with some degree of periodicity. However, other energy models for aperiodic use cases are equally possible. The only requirement is embedding the controls to the configuration of parameters for future energy prediction in \fref{sec:mpc}{Section}. A possible model is a differential linear aperiodic model built via regressional analysis, mapping a change in the schedule to, e.g., the slope of the regression. Other examples are models that incorporate robots dynamics, that map the energy to turns, etc.

% costs different from energy OK

\fref{sec:2pbs}{Section} mentions the optimal configuration in \frefeq{eq:disc-ocp-output-mpc} with costs other than the energy. For instance, it might be required to cover a given space with the fastest tour possible. Or, similarly, selectively prioritize some computations over the others. These considerations were not covered by \fref{algo:repla}{Algorithm}, which merely selects the best possible schedule w.r.t. to the given cost in \frefeq{eq:insta-cost-mpc}. In a prioritized setting, it might be then possible to utilize a stage-dependent priority list: let's call it $\mathbf{r}_i\in(0,1]\subset\mathbb{R}^n$ where $n$ is given in \fref{sec:nom-cont}{Section}. The list can be used in the cost with the parameters,  prioritizing specific computations for particular stages. In this setting, it might be possible to trade-off other non-functional properties, such as security involved in the TeamPlay\findex{TeamPlay} project~\citep{teamplay} that funded a considerable part of this work, by, e.g., increasing the latter at given ``security demanding'' stages.

% 3D generalization for the path functions OK

The construct of path functions in \fref{sec:path-functions}{Section} involves aerial coverage flying at a given altitude $h$. A more complex coverage planning with, for example, the aerial robot covering a building requires further investigation. In this latter case, \fref{def:paths}{Definition} might be extended to functions $\mathbb{R}^3\rightarrow\mathbb{R}$, expressing simply a 3D shape of, e.g., the building to cover with a cylinder. The guidance in \fref{sec:gvf}{Section} would then adopt different strategies. For example, the introduction of multiple path functions in 2D discretely ``slicing'' the cylinder, or the development of the theory of vector fields further with, e.g., the magnitude of the gradient in \frefeq{eq:grady}. The varying quality of the coverage can be achieved by changing the distance between the ``slicing'' functions.

% extension of to other variable coverage motions OK
% consider different i (evolve the paths) in the best configuration MPC algorithm OK

Future directions involve the extension of the coverage motion, the study of additional motions and their trade-offs, and the shortcomings of \fref{algo:repla}{Algorithm}. The derivation of the optimal parameters configuration happens for the current stage $i$ and assumes future stages evolve according to the energy model and the observation in \fref{sec:nom-cont}{Section}. The energy of complex plans might, however, change completely at each stage and require a stage-dependent energy model. \fref{algo:repla}{Algorithm} would necessitate a trajectory to map the time to the stage for energy predictions in this latter stage-dependent energy model.

% proper obstacles (no-flight zones rather than no-interest zones) OK
% more accurate battery modeling OK

In \fref{sec:algo}{Section}, we scheduled in a NIZ-aware fashion overflying, e.g., supporting infrastructure, without computing. The derivation of a coverage plan accounting for both NIZs and no-flight zones (\Gls{acr:nfz}s) requires further analysis. Other than for the physical obstacles in form of, e.g., tall buildings, it might be indeed more convenient to avoid flying over a NIZ energy-wise. For instance, if the NIZ is attached to one of the vertices $v$. For some cases, it might be then advantageous to consider complex battery models, including details such as different battery chemistries, state of health, temperature, or C-rate. Additional rigor might lead to better modeling with other equivalent electrical circuits models such as the Thevenin-based model~\citep{chen2006accurate,hasan2018exogenous,hinz2019comparison,mousavi2014various,zhang2018online,salameh1992mathematical} outlined in \fref{sec:batmod-circuit}{Section}.

% other planning approaches rather than only coverage OK
% multi robot strategies OK

Further possible development is that of the derivation of an approach for planning-scheduling energy awareness with multiple robots. Here a swarm of aerial robots might be covering the agricultural field with a centralized or distributed system handling overall multi-agent planning-scheduling. The system might then integrate \fref{algo:repla}{Algorithm} for coverage replanning by deciding a piece-wise coverage. It might even distribute the computations to optimize the battery utilization of each agent while preserving given use case constraints.
Integration of various approaches might lead to additional new developments; for instance, considering the frequency and voltage from other planning-scheduling literature~\citep{zhang2007low,brateman2006energy} in \fref{algo:repla}{Algorithm}.

These additional studies will possibly further provide insights into planning-scheduling energy awareness and expedite the research foundations of varying both aspects together, bridging low power computing and mobile robotics domains.


%%%%%%%%%%%%%%%%%%%%
\section{Conclusion}
\label{sec:conc-conc}

%(a) how to model the computations energy to predict the impact of any possible schedule on the computing hardware’s power? 
%(b) How to model the motion energy to predict the impact of any possible variation of motion on the aerial robot’s energy? 
%(c) How to merge these two points with a cohesive model? 
%(d) How to derive an optimal configuration of the schedule and motion?

%Aerial robots are an ideal instance of energy-constrained systems, benefitting the simultaneous planning-scheduling energy awareness.

In conclusion, even though planning-scheduling energy awareness is a relatively recent research direction, there are some contributions in the wider mobile robotics research literature. Initial studies date back to 2000--2010 and focus on varying computational aspects such as frequency and voltage, along with motional aspects such as velocity or selection of peripherals. Studies in the following decade 2010--2020, especially in the second half, dig further into computational aspects proposing various techniques for simultaneous planning-scheduling, emphasizing the possible savings. Past approaches, however, do not account for aerial robots. With this work, on the contrary, we emphasize planning-scheduling energy awareness for this latter class of mobile robotics platforms. We further exploit the battery as an energy source and the possibility of in-flight online replanning.

We derive an optimal configuration of both the path and computations, running on heterogeneous computing hardware mounted onto the aerial robot. We use a sub-class of motion planning where the aerial robot covers each point in a given space, varying the quality of the coverage and ground hazards detections in case of, e.g., sudden battery drops, adverse atmospheric conditions, etc. We observe the effect of replanning, with the aerial robot completing the coverage under adversities or exploiting the energy budget with ideal conditions in the function of the available battery SoC. We can back the statement in the introduction: aerial robots {\itshape are an ideal instance} of an energy-constrained system benefitting from planning-scheduling trade-offs, answering the research question {\itshape (d)} in \fref{sec:methodology}{Section}. 

The derivation of such a conclusion required multiple steps. Firstly, it was necessary to predict the energy of future instances of computations, which we addressed with the \powprof{} modeling and profiling tool, answering the research question {\itshape (a)}. Knowing only the energy evolutions of computations is not enough. We thus derived a differential periodic energy model that exploits coverage characteristics and empirical observations, answering the research question {\itshape (b)}, and that predicts energy of variations of schedules and paths, answering {\itshape (c)}. 

