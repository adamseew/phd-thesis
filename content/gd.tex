
%%%%%%%%%%%%%%
%            %
% Guidance   %
%            %
\chapter{\color{red}Guidance}
\label{cp:gd}

%\begin{highlight}
%    \begin{st}
%        General structure with just some dummy text.
%    \end{st} 
%\end{highlight}

%\lettrine{A}{a}


\section{\color{red}Vector Fields for Guidance}

 The use of vector fields for guidance is widely discussed in the literature~\citep{lindemann2005smoothly,gonccalves2010vector,panagou2014motion,zhou2014vector,kapitanyuk2017guiding,de2017guidance}.

 We derive the new position $\mathbf{p}_{k+1}$ computing the vector field 
 \begin{equation}
   \nabla\varphi_i:=\begin{bmatrix}\partial\varphi_i/\partial x \\ \partial\varphi_i/\partial y\end{bmatrix},  
 \end{equation}
 and the direction to follow in the form of velocity vector~\cite{de2017guidance}
 \begin{equation}\label{eq:pd}
   \dot{\mathbf{p}}_d(\mathbf{p}_k):=E\nabla\varphi_i-k_e\varphi_i\nabla\varphi_i,
 \end{equation}
 where $E$ specifies the rotation (it influence the tracking direction). For instance
 \begin{equation}
   E=\begin{bmatrix}
     0&1\\-1&0
   \end{bmatrix},
 \end{equation}
 is the counter clockwise direction, $-E$ the clockwise direction. 
 
 $k_e\in\mathbb{R}_{\geq 0}$ is the gain to adjusts the speed of convergence. The direction the velocity vector $\dot{\mathbf{p}}_d$ is pointing at is generally different from the course heading $\dot{\mathbf{p}}$ due to the atmospheric interferences (wind $w\in\mathbb{R}$ in the top of \fref{fig:tee1}{Figure}).

\section{\color{red}Derivation of the Guidance Action}

\subsection{\color{red}Motion simulations}

\subsection{\color{red}Energy simulations}


\section{\color{red}Alteration of the Path}


\section{\color{red}Results}


\section{\color{red}Summary}

