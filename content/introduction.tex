
%
%%%%%%%%%%%%%%%%%%
%                %
% Introduction   %
%                %
%%%%%%%%%%%%%%%%%%
%
% Brief abstract: introduction to the rest of the work
%
% Completion (1-10): 9.3
% Missing: Methodology is missing. 
% Proofreading: 7/10/21
%
\chapter{Introduction}
\label{cp:intro}

\begin{chapquote}{\cite{wang2017curvature}}
  ``Fixed-wing[s] [...] tend to be more stable in the air in the face of both piloting and technical errors as they have natural gliding capabilities even without power, and they are able to travel longer distances on less power.''
\end{chapquote}

\vspace*{1em}

\lettrine{M}{obile robots} have the ability to move~\citep{corke2017robotics} and eventually sense and interact with the surrounding environment. Most of these robots are power-demanding devices constrained by battery limitations. Although common in many areas, such limitations are critical in mobile robots' design and development--they might influence the level of autonomy\findex{autonomy}, in turn, expected to increase in the foreseeable future~\citep{fisher2013verifying}.

To move mobile robots combine and interpret the data from multiple components~\citep{mei2006deployment} via, e.g., various algorithms for perception\findex{perception algorithms} and planning\findex{planning algorithms}, running on energy-demanding heterogeneous computing hardware\findex{heterogeneous computing hardware}. For instance, with a motion planning algorithm\findex{motion planning}, a mobile robot converts and combines a human-level plan with sensing into some machine-level motion primitives~\citep{lavalle2006planning}\findex{motion primitives}. There are numerous planning algorithms applied to a variety of robots, yet some approaches are less explored than others. Planning an energy-aware path with a power-saving scheduling policy\findex{scheduling} for the computing hardware is one of such underrepresented approaches in the robotics research literature~\citep{sudhakar2020balancing,lahijanian2018resource,ondruska2015scheduled,brateman2006energy}. Past studies well cover one of these topics, but the interactions between the two are mostly unexplored~\citep{brateman2006energy}. Some generate an energy-optimized path, but the computations\findex{computations} energy might equal the motion in some instances of low-energy mobile robots~\citep{sudhakar2020balancing}. Due to the recent advancements in the computational capabilities of heterogeneous computing hardware, such as the introduction of powerful portable GPUs\findex{GPU}~\citep{rizvi2017general}, the use of computations is then further on the rise~\citep{abramov2012real,satria2016real,jaramillo2019visual}. Others provide a power-saving scheduling policy, yet, moving a mobile robot requires considerable energy expenditure over mere computations~\citep{mei2004energy,mei2005case}.

In this work, we focus on both, investigating their tradeoffs. Within motion planning itself, we focus on a specific category. In autonomous scenarios, it is often required to explore every location in space~\citep{cao1988region,choset2000exact,arkin2001optimal,arkin2005optimal}, a problem solved by coverage planning~\citep{choset2001coverage,galceran2013survey}\findex{coverage path planning}. We thus showcase the interactions between planning-scheduling on a coverage planning algorithm--we cover a space scheduling high-level task for, e.g., perception, in an elemental region\findex{elemental region}. We demonstrate the approach both in simulation and empirically, and we focus on aerial mobile robots. Although these share with the broader class of mobile robots stringent battery limitations, they have further complications. In a conventional setup, for instance, it would be required landing to replace or recharge the battery. They thus, perhaps, provide an ideal instance of energy-constrained systems that would benefit from planning-scheduling energy awareness. We will see that we can back such a statement with observations of actual flights failure reductions later with this work in \fref{cp:res}{Chapter}.

There are numerous autonomous use cases involving aerial robots, such as precision agriculture, search and rescue\findex{search and rescue}, payload delivery\findex{payload delivery}, transportation\findex{transportation}, and many others. We formulate the problem later in \fref{cp:pb}{Chapter} having in mind a precision agriculture use case of an autonomous fixed-wing aerial robot flying over an agricultural field with little human input.
\begin{figure}[t]
  \centering
  \includegraphics[width=.7\textwidth]{pictures/photo}
  \caption[Opterra fixed-wing aerial robot]{Opterra fixed-wing aerial robot employed for precision agriculture {\scriptsize(photo credit: Amit Ferencz Appel)}.}
  \label{fig:opterra}
\end{figure}
Precision agriculture\findex{precision agriculture} is indeed often put into practice~\citep{hajjaj2014review} with ground mobile robots used for harvesting~\citep{qingchun2012study,dong2011development, de2011design, aljanobi2010setup, li2008analysis, edan2000robotic}, and aerial robots for preventing damage and ensuring better crop quality~\citep{puri2017agriculture, daponte2019review}. We investigate different physical aerial robotics platforms in this work but we derive most results with the Opterra fixed-wing aerial robot\findex{Opterra fixed-wing aerial robot}~\citep{opterra} adapted for precision agriculture. The aerial robot is in \fref{fig:opterra}{Figure}.

In the remainder of the chapter, and before we introduce the problem in \fref{cp:pb}{Chapter}, we briefly investigate the evolution of the field of aerial robotics in \fref{sec:history}{Section}. We then analyze the main aerial robots available today and how they apply to energy-aware planning-scheduling in \fref{sec:aerial-robo-types}{Section}, justifying the fixed-wings. We then provide some further motivation in \fref{sec:motivation}{Section} and the outline of the approach in \fref{sec:outline}{Section}. In \fref{sec:applics}{Sections}\fref{sec:contribs}{--\hspace*{-.8ex}}, we discuss the applications, methodology, and our overall contribution. Finally, we provide the structure of the remaining chapters in \fref{sec:structure}{Section}.

This chapter connects to the remainder of this work as follows. Here we introduce and motivate the energy-aware coverage planning and scheduling for autonomous aerial robots. We formalize the planning problem in \fref{cp:pb}{Chapter} and describe the derivation of an energy model to predict future energy consumption in \fref{cp:model}{Chapter}. We derive an optimal configuration of the path and computations using data-driven control\findex{data-driven control} and other modern optimal control techniques in \fref{cp:dyn}{Chapter}. In the same chapter, we provide some details for guidance, which physically moves the robot. We use the configuration for coverage planning and scheduling of the aerial robot experimentally in \fref{cp:res}{Chapter}. Moreover, we discuss previous approaches in the literature that are relevant to our work in \fref{cp:soa}{Chapter}.


%%%%%%%%%%%%%%%%%%%%%%%%%%%%%%%%%%%%%%%%%%%
\section{From UAVs to Modern Aerial Robots}
\label{sec:history}

Modern aerial robots are a valuable tool in robotic research and aerospace. They have different names in the literature, including unmanned aerial vehicles\findex{unmanned aerial vehicles} (\Gls{acr:uav}s)\footnote{The term unmanned is sometimes replaced by uninhabited}, unmanned aerial systems\findex{unmanned aerial systems} (\Gls{acr:uas}s)\footnote{UAS often denotes the entire infrastructure of unmanned flight in the aerospace jargon}, flying robots, or drones. Usually, we refer to drones, UAVs, and UASs when these systems are semi-autonomous or operated from the ground, whereas aerial or flying robots when they have advanced levels of autonomy~\citep{siciliano2016springer}. Nevertheless, all these systems have basic autonomous features such as position holding implemented with, e.g., the global navigation satellite system\findex{global navigation satellite system} (\Gls{acr:gnss}), altitude holding with a barometer\findex{barometer}, and leveling with the inertial measurement unit\findex{inertial measurement unit} (\Gls{acr:imu}).

\begin{figure}[t]
  \centering
  \includegraphics[width=.7\textwidth]{pictures/HA-NH-JA-19_1}
  \caption[Hewitt-Sperry Automatic Airplane, first unmanned flying machine]{The Hewitt-Sperry Automatic Airplane, also denominated ``flying bomb'', was developed in WWI and represents the first instance of a \Gls{acr:uav} {\scriptsize(photo credit: United States Naval Institute)}.}   
  \label{fig:hewitt-sperry}
\end{figure}
Unmanned (or uninhabited) flight has more than a century of developments~\citep{siciliano2016springer}. The origin of the field of aerial robotics, which deals with the design and development of aerial robots, dates back to the first guided missiles. Hewitt-Sperry Automatic Airplane\findex{Hewitt-Sperry automatic airplane} in \fref{fig:hewitt-sperry}{Figure}, also denominated ``flying bomb'' and deployed in 1917 during World War~I\findex{World War I}~(WWI)~\citep{keane2013brief,valavanis2015handbook}, is often referred to as the first unmanned flying machine. Developed 14 years after the first heavier-than-air flight in history--demonstrated on December 17, 1903, with the Wright Flyer I by Wilbur and Orville Wright\findex{Wright brother} (or the Wright brothers)--it used a gyroscope\findex{gyroscope} invented shortly before by Elmer Sperry~\citep{keane2013brief}, alike modern aerial robots. The device was mechanically connected to the control surfaces, successfully implementing a control feedback loop~\citep{siciliano2016springer}.

In their early days, these first flying machines were referred to as remotely piloted vehicles\findex{remotely piloted vehicles} (\Gls{acr:rpv}s)~\citep{anderson2005introduction}. Many instances from WWI were designed for military purposes. In the 1950s, United States used a remotely controlled vehicle, the Ryan Firebee\findex{Ryan Firebee}, for reconnaissance in Vietnam, and Israel was the first to use an \Gls{acr:rpv} in a combat situation~\citep{anderson2005introduction}. Other instances include the V-1 flying bomb~\findex{V-1 flying bomb} from 1944 (deployed by the unified armed forces of nazi Germany) and the Lockheed D-21~\findex{Lockheed D-21} from 1962 (deployed by the United States Air Force). The introduction of the global positioning system\findex{global positioning system} (\Gls{acr:gps}) at the end of 1970 allowed some more recent applications, including surveillance. Later, flying machines were integrated with cameras and other sensors~\citep{siciliano2016springer}, in what are the modern aerial robots.

Unmanned flight has entered many civilian applications in recent years~\citep{gonzalez2017unmanned}. Aerial robots are used increasingly in remote sensing\findex{remote sensing}~\citep{colomina2014unmanned,noor2018remote,tang2015drone,milas2018drones}, surveillance\findex{surveillance}~\citep{acevedo2014one,ramasamy2017heuristic,basilico2015deploying,paucar2018use,burkle2009collaborating}, meteorology\findex{meteorology}~\citep{renzaglia2016monitoring,schuyler2019using}, search and rescue\findex{search and rescue}~\citep{hayat2017multi,pensieri2020drones,karaca2018potential,cui2015drones,seguin2018unmanned}, precision agriculture~\citep{popovic2017online,sa2018weednet,lottes2017uav,daponte2019review,puri2017agriculture}, transportation, and payload delivery\findex{payload delivery}~\citep{kellermann2020drones}. The former four categories fall into reconnaissance, surveillance, and target acquisition\findex{reconnaissance, surveillance, and target acquisition} (\Gls{acr:rsta}) and do not require advanced autonomy, while precision agriculture, transportation, and payload delivery utilize a certain extent of computational intelligence~\citep{siciliano2016springer}. These systems handle unexplored terrain with little interaction, whereas the past UAVs were operated mainly by a human operator~\citep{siciliano2016springer}. Instances autonomously adapt and possibly interact in a wide variety of environmental conditions.

\begin{figure}[t]
  \centering
  \includegraphics[width=.7\textwidth]{pictures/jpegPIA24550}
  \caption[NASA's Ingenuity Mars Helicopter]{NASA's Ingenuity Mars Helicopter. A rotary-wing coax aerial robot that achieved the first powered, controlled flight on another planet on April 19, 2021. The project is solely a demonstration of technology {\scriptsize(photo credit: NASA Jet Propulsion Laboratory)}.}   
  \label{fig:ingenuity}
\end{figure}
In summary, aerial robots have a relatively recent past. Some initial experiments of unmanned flights were performed shortly after the first heavier-than-air manned and powered flight. These initial experiments were often developed for military purposes, whereas modern aerial robots fly in a broad range of civilian applications. Aerial robots are to grow significantly in numerous areas in and out of robotics research ranging from agriculture to planetary exploration. For the latter, \fref{fig:ingenuity}{Figure} shows NASA's Ingenuity Mars Helicopter\findex{Ingenuity Mars Helicopter}\findex{NASA}. A small coaxial aerial robot\findex{coaxial aerial robot}\findex{Mars} that performed the first powered, controlled flight on another planet on April 19, 2021. Aerial robots for planetary exploration are to be further deployed in future explorations endeavors, for instance, to study Saturn's moon Titan~\citep{voosen2019nasa}\findex{Saturn}\findex{Titan}.


%%%%%%%%%%%%%%%%%%%%%%%%%%%%%%%%%%%%%%%%%
\section{Common Classes of Aerial Robots}
\label{sec:aerial-robo-types}

There are numerous different types of aerial robots. We briefly investigate the most studied classes in the robotics literature and relate them to the energy-aware planning-scheduling in this work. The two most generic classes are heavier-than-air and lighter-than-air aerial robots. Heavier-than-air aerial robots\findex{heavier-than-air aerial robots} are divided into fixed- and rotary-wings~\citep{siciliano2016springer}, and some recent contributions in bio-inspired robotics investigate flapping-wings~\citep{floreano2015science}\findex{flapping-wings}. 

Rotary-wing aerial robots\findex{rotary-wings} are highly maneuverable and can perform stationary vertical flight (commonly referred to as hovering\findex{hovering})~\citep{siciliano2016springer}. These systems can be classified into further categories, which include multirotors\findex{multirotors} (such as quadrotors\findex{quadrotors} or quadcopters\findex{quadcopters}, hexacopters\findex{hexacopters}, and octocopters\findex{octocopters}), conventional helicopters\findex{helicopters} (these have one main and one tail rotor), and coaxes\findex{coax} (these have counter-rotating coaxial rotors)~\citep{corke2017robotics}. Examples of quadrotors are DJI Mavic Mini in \sref{lab:mavic} and DJI Phantom 4 in \sref{lab:phantom} in \fref{fig:robots-vs-power}{Figure}. In the same figure, DJI Agras T16 in \sref{lab:agras} and DJI Matrice 600 in \sref{lab:matrice} are hexacopters.

Fixed-wing aerial robots\findex{fixed-wings} have wings to provide the lift, some control surfaces for maneuvering, and propellers for the forward thrust; a shared principle with a common passenger aircraft~\citep{corke2017robotics}. An example constitutes the Opterra adapted for precision agriculture in \fref{fig:opterra}{Figure}, Cumulus in \sref{lab:cumulus}, Ebee in \sref{lab:ebee}, and Penguin BE in \sref{lab:penguin}~\citep{haugen2016monitoring} in \fref{fig:robots-vs-power}{Figure}. Examples of flapping-wings are Delfly II in \sref{lab:delfly}~\citep{percin2012flow} and Nano-Hummingbird in \sref{lab:nano} in \fref{fig:robots-vs-power}{Figure}. Instances of lighter-than-air aerial robots\findex{lighter-than-air aerial robots} are blimps\findex{blimps} (or non-rigid airships). They usually rely on gas, e.g., helium\findex{helium} enclosed in a protected envelope~\citep{burri2013design}, to generate the lifting force\findex{lift}~\citep{fui2017recent}. An omnidirectional spherical blimp is in \fref{fig:skye-blimp}{Figure}. Blimps are similar to balloons but provide basic maneuverability, whereas a balloon can control merely the altitude~\citep{colombatti2011lighter}.
\begin{figure}[t]
  \centering
  \includegraphics[width=.7\textwidth]{pictures/IMG_2612}
  \caption[Skye, an omnidirectional spherical blimp]{Skye, an omnidirectional spherical blimp developed by ETH Z\"urich for entertainment purposes. It has a camera system and combines the energy-efficient flight of a blimp with the characteristics of a quadrotor {\scriptsize(photo credit: ETH Z\"urich)}.}   
  \label{fig:skye-blimp}
\end{figure}
Some other classifications found in aerial robotics literature use size and maneuverability and include classes such as micro aerial vehicles (\Gls{acr:mav}s)\findex{micro aerial vehicles}, vertical take-off and landing (\Gls{acr:vtol}s)\findex{vertical take-off and landing} aerial robots, and others. The former are aerial robots with all dimensions lower than 15 centimeters. The latter are aerial robots flying in a fixed-wing configuration except for taking-off and landing; in these two configurations, they use thrust\findex{thrust} from rotors rather than lift from wings. 

Among the classes in this section, rotary-wings are the most maneuverable, and lighter-than-air aerial robots are the least. These, however, have the highest flight time followed by fixed-wing aerial robots. Mixed configurations, such as VTOLs, fall into the intersection of rotary- and fixed-wings for what concerns maneuverability and flight time~\citep{siciliano2016springer}. The energy requirements are critical for all aerial robots, but the difference of motion and computations energy (\Gls{acr:mace}) varies greatly. It is highest in rotary-wings and lowest in lighter-than-air aerial robots in \fref{fig:robots-vs-power}{Figure}. Planning-scheduling would thus rely on both for the fixed-wings robots energy-wise while relying almost exclusively on planning for rotary-wing aerial robots. In the former, M\&CE is close to zero. In the latter, it is usually high except for energy-optimized designs such as some rotary-wing MAVs. Hypothetically, in lighter-than-air aerial robots, M\&CE might be negative; the planning-scheduling would rely heavily on scheduling.

The classification that we proposed here serves the scopes of our work. There are similar efforts for grouping aerial robots by size and propulsion technology~\citep{hoffer2014survey,cabreira2019survey}, but other classifications are possible as well, e.g., categorization by altitude~\citep{watts2012unmanned}.


%%%%%%%%%%%%%%%%%%%%
\section{Motivation}
\label{sec:motivation}

\begin{figure}[p!]
  \centering
  \footnotesize\fontfamily{phv}\selectfont
  \input{figures/robots-vs-power.tikz}
  \vspace*{42pt}
  \caption[Different aerial robots in relation to the power, flight time, and M\&CE]{Different aerial robots in relation to the power, flight time, and M\&CE with a hypothetical fixed costs for computations energy. The power is expressed using a logarithmic scale. Heavier-than-air aerial robots \sref{lab:cumulus}{}--\sref{lab:nano}{} include flapping-wings \sref{lab:delfly}{},~\sref{lab:nano}{} and have a negative M\&CE (computations scheduling is to be accounted for most in planning), whereas rotary-wings have a positive M\&CE \sref{lab:matrice}{},~\sref{lab:agras}{} (path planning is to be accounted for most in dynamic planning). Some smaller rotary-wings have an M\&CE closer to zero \sref{lab:mavic}{},~\sref{lab:phantom}{}. In general, rotary-wings have a short flight time. Fixed-wings \sref{lab:cumulus}{}--\sref{lab:opterra}{} have a considerably longer flight time and M\&CE close to zero (computations scheduling and path planning have both to be accounted for in dynamic planning). Lighter-than-air aerial robots \sref{lab:skye}{} have hypothetically a relatively long flight time and lower than zero M\&CE {\scriptsize(photos credit: \srefsmaller{lab:cumulus} to Sky-Watch, \srefsmaller{lab:matrice} to Rise Above, \srefsmaller{lab:agras} to Aeromotus, \srefsmaller{lab:mavic} to Digital Photography Review, \srefsmaller{lab:phantom} to ePHOTOzine, and \srefsmaller{lab:nano} to DARPA)}.}
  \label{fig:robots-vs-power}
\end{figure}

Many applications involving aerial robots have strict battery constraints. It is a common problem of most mobile robots~\citep{mei2006energy}, yet, aerial robots are particularly affected; indeed, the availability of the power source might influence their autonomy. An instance is an aerial robot autonomously inspecting a given space by, e.g., detecting ground patterns and notifying other ground-based actors with little human interaction in a precision agriculture use case. For such and many others, aerial robots often rely on computing hardware along with a microcontroller~\citep{dharmadhikari2020motion,william2019aerial,papachristos2015aerial,holper2017cyber}. Computing hardware provides autonomous capabilities and planning, whereas microcontroller\findex{microcontroller} runs motion primitives by directly interfacing actuators (such as servos\findex{servo} for the control surfaces\findex{control surface}) and motors\findex{motor}~\citep{mei2005case}.

\subsection{Path planning and computations scheduling}

It is uncommon to find a ready-to-use solution for both planning the path and scheduling the computations of these systems in an energy-aware fashion~\citep{brateman2006energy,sudhakar2020balancing}. Yet, the energy requirements of the computing hardware are a further complication. It might be potentially advantageous to schedule the computations on the computing hardware and simultaneously to plan the path rather than solving these two separately~\citep{lahijanian2018resource,ondruska2015scheduled} by, e.g., optimizations in terms of the battery state of charge (\Gls{acr:soc}). 

For certain classes of aerial robots with M\&CE close (or lower than) zero, the autonomy can directly influence the battery state. For these classes, it is desirable to reschedule the computations energy-wise in-flight during a motion energy-demanding phase. For instance, a fixed-wing aerial robot might be flying headwinds (with the wind vector parallel and opposite to the direction of motion) and utilizing more energy than planned. It would be of advantage to reschedule the tasks accordingly and compensate for the increment in motion energy. During the same flight, the wind direction might suddenly change. The fixed-wing craft, now flying tailwinds, requires less motion energy. It could then potentially increase the level of computations by rescheduling the tasks. Later in the flight, the battery might be subject to sudden drops due to, e.g., temperature changes, requiring replanning again by, for instance, shortening the path. Planning-scheduling altogether in all these cases is, perhaps, the most desirable course of action.

In \fref{fig:robots-vs-power}{Figure}, we show the M\&CE against the flight time of different aerial robots. We observe that fixed-wings are the aerial robots that would advantage the most from simultaneous planning-scheduling. They have an M\&CE close to zero and a relatively long flight time. Although some rotary-wings have M\&CE also close to zero, their flight time is generally short. Flapping-wings and lighter-than-air aerial robots require considerably less energy for the motion and have a negative M\&CE.

\subsection{Objective}
\label{sec:objective}

In the remainder of this work, we refer to computational tasks that can be scheduled in an energy-aware fashion as computations, opposed to others with no significant effect on energy consumption. We assume the aerial robot runs the computations on the heterogeneous computing hardware. As an example, we refer to an aerial robot in a precision agriculture use case in \fref{fig:opterra}{Figure}, doing coverage planning while detecting ground hazards. 
\begin{figure}[h!]
  \centering
  
\def \globalscale {1.000000}
\begin{tikzpicture}[y=0.80pt, x=0.80pt, yscale=-.7*\globalscale, xscale=.7*\globalscale, inner sep=0pt, outer sep=0pt]

\path[cm={{1.0,0.0,0.0,1.0,(0,0)}}] (0.0000,0.0000) node[below right] () {\includegraphics[width=3.122in]{figures/source/out30(1)}};

\path[fill=foo,line join=round,fill opacity=0.35,line width=0.256pt] (0.0000,247.7590) -- (227.8430,146.1690) -- (403.2000,150.1500) -- (403.2000,273.8490) -- (0.0000,247.7590) -- cycle;

\path[draw=white,line join=round,line width=0.512pt] (19.6911,244.5640) -- (56.0711,227.9040);



\path[draw=white,line join=round,line width=0.512pt] (19.8503,205.9030) -- (19.8504,244.6440) -- (67.5301,247.1570);



\path[fill=white,line join=round,line width=0.160pt] (63.6000,244.7010) -- (65.5307,247.0370) -- (63.4689,249.0560) -- (69.3046,247.0560) -- (63.6000,244.7010) -- cycle;



\path[fill=white,line join=round,line width=0.160pt] (52.5130,227.1350) -- (55.3596,228.1760) -- (54.6000,230.9590) -- (58.6260,226.2860) -- (52.5130,227.1350) -- cycle;



\path[fill=white,line join=round,line width=0.160pt] (17.6441,209.5580) -- (19.8958,207.5300) -- (22.0002,209.5030) -- (19.7535,203.7580) -- (17.6441,209.5580) -- cycle;



\path[cm={{1.0,0.0,0.0,1.0,(13.0,267.0)}}] (0.0000,0.0000) node[above right] () {\color{white}$\mathcal{O}_W$};



\path[cm={{1.0,0.0,0.0,1.0,(69.0,259.0)}}] (0.0000,0.0000) node[above right] () {\color{white}$x$};



\path[cm={{1.0,0.0,0.0,1.0,(50.0,220.0)}}] (0.0000,0.0000) node[above right] () {\color{white}$y$};



\path[cm={{1.0,0.0,0.0,1.0,(10.0,203.0)}}] (0.0000,0.0000) node[above right] () {\color{white}$z$};

\end{tikzpicture}


  \caption[The coverage problem in a precision agriculture scenario]{The coverage problem in a precision agriculture scenario. The aerial robot covers an agricultural field that forms a polygon (blue/transparent area in the frame) and runs some autonomous tasks {\scriptsize(photo credit: Amit Ferencz Appel)}.}
  \label{fig:plot2}
\end{figure}
In abstract terms, the field is a polygon in \fref{fig:plot2}{Figure}, and the aerial robot detects some patterns using a convolutional neural network (\Gls{acr:cnn})\findex{convolutional neural network}. The aerial robot further communicates the position of a detected pattern to other, ground-based actors. Detecting the patterns usually involves heterogeneous computing elements (GPU\findex{GPU} and/or multi-core CPU\findex{CPU}\findex{multi-core CPU}) and consumes a significant amount of energy contrary to, e.g., communication. Detection is a computation, communication a task.

We are interested in the energy optimization of the path and schedule under uncertainty (atmospheric interferences) in-flight. Unlike most of the past literature, the approach that we propose plans these two aspects simultaneously. Such planning would find optimal tradeoffs\findex{tradeoffs} between the path, computations, and energy requirements. Current generic solutions for, e.g., aerial robots coverage planning, do not investigate the two aspects simultaneously, nor are they energy-aware. They are often semi-autonomous: the path and computations are static and usually defined using planning software~\citep{daponte2019review} from existing algorithms~\citep{choset2001coverage,galceran2013survey}, with instances including popular flight controllers~\citep{px4,papa}. This state of practice has prompted us to investigate the possible interaction between planning-scheduling applied to coverage path planning (\Gls{acr:cpp})\findex{coverage path planning}. In the remainder, we will gradually build an energy-aware coverage planning and scheduling approach for autonomous aerial robots. It plans and schedules altogether while the aerial robot flies and its batteries drains under a variety of conditions.


%%%%%%%%%%%%%%%%%%%%%%%%%%%%%%%%%
\section{Outline of the Approach}
\label{sec:outline}

For planning-scheduling altogether, we need some information on the use case for both the path and computations. Path-wise, these include data on the coverage area, such as a description of an equivalent polygon. We provide detailed ways of defining the problem later in \fref{cp:pb}{Chapter}. From the polygon, a coverage algorithm generates a plan composed of multiple stages, later eventually replanned by a replanning algorithm in case of, e.g., sudden battery drops. At each stage, the aerial robot flies a path and executes some computations. We use the concept of different stages to model complex paths, e.g., multiple circles and lines form the overall coverage. The robot switches between the paths in the proximity of specific triggering points. The plan further contains some additional parameters to alter the path and computations along with an energy budget. We will see the plan later in \fref{sec:flight-plan}{Section}. The alterations are bounded. There are path constraint sets that bound the path alterations and computations constraint sets, one per each computation, that bound computations alterations. The approach guides the aerial robot with a vector field-based gradient descent algorithm in \fref{sec:gvf}{Section}\findex{vector field}\findex{gradient descent}. We will see these algorithms in \fref{cp:dyn}{Chapter} and the building constructs (plan, stages, parameters, paths, constraints, triggering points) in \fref{cp:pb}{Chapter}.

Computation-wise, we need an approach to specify the computations and quantify their energy contribution. We cover these aspects in \fref{cp:model}{Chapter}, along with the motion and battery energy models. The approach relies on \powprof{}~\citep{powprofiler}, a tool that we introduce in \fref{sec:comp-ener-model}{Section}, part of our early studies~\citep{seewald2019component,seewald2019coarse}. The tool models the power, energy, and battery SoC of the computations and is embedded in the future energy estimations of the aerial robot in \fref{sec:deriv}{Section}. To this end, we empirically derive and formally prove a periodic differential energy model that accounts for the uncertainty. We use Fourier analysis to derive the model and state estimation to address the uncertainty. Periodicity is due to the periodic patterns in the coverage plan, as we will see in \fref{cp:model}{Chapter} and \fref{cp:res}{Chapter}. Indeed, in the coverage, as well as some other autonomous scenarios, the mobile robot often iterates over a set of tasks and paths~\citep{seewald2020mechanical,seewald202Xenergy}. Given that the plan is periodic, we expect the energy consumption to evolve (approximately) periodically. We will back the statement with both experimental and realistic simulated findings in~\fref{cp:res}{Chapter}.

Once we have a plan and all the components to model the energy and battery, we can (re)plan-schedule the coverage online in-flight, aided by modern optimal control techniques, e.g.,  state estimation and model predictive control (MPC)~\citep{rawlings2017model,simon2006optimal} in \fref{cp:dyn}{Chapter}. The control is data-driven: energy sensor data estimates some coefficients of the model to predict the future energy consumption with uncertainty while obeying the energy budget--the battery capacity and other battery parameters. Our goal is to complete the plan with the highest possible parameters configuration.


%%%%%%%%%%%%%%%%%%%%%%
\section{Applications}
\label{sec:applics}

In the remainder of this work, we focus on the precision agriculture use case where we plan the coverage and schedule hazards detection. Nonetheless, the approach works with other potential use cases. Earlier collaborations between consortium members of the TeamPlay project~\citep{teamplay}, funding this work has led to a search and rescue use case, where an aerial robot detects vessels in an offshore area, eventually planning-scheduling of the search pattern and the detection rates. We briefly detailed the scenario in our earlier study~\citep{seewald2019coarse}, which has been since extended with a deadline guarantee scheduling policy~\citep{rouxel2020prego}. We have then attempted other use cases, e.g., mapping in the planetary exploration context, where we hypothesized the possibility of scheduling navigation~\citep{seewald2020beyond}. Indeed an earlier study proposes a similar technique, scheduling perception~\citep{ondruska2015scheduled}, and perhaps further motivating our analysis. Simulated use cases are also possible. In an early study~\citep{zamanakos2020energy} of a use case in agricultural safety, we followed a simulated agricultural vehicle by varying the tracking algorithm. There are multiple possibilities to apply our approach to a broad range of real-world and simulated use cases, arising in many different fields in and outside of basic robotics research. Generally, the planning-scheduling energy awareness in this work applies to modern aerial robots with a certain degree of autonomy, whereby the robot performs at least a predefined set of tasks over a given space. Indeed most of the guidance in \fref{cp:dyn}{Chapter} works well for aerial robots, yet, one can adapt our work to other mobile robots with energy constraints. We discuss applications out of the aerial robotics domain further in \fref{cp:conc}{Chapter}. We expect the approach to be most relevant to energy-efficient mobile robots, with the best outcomes for M\&CE close to zero (the motion and computations energy contributions are similar). Such conclusion is shared with other studies researching planning-scheduling energy awareness~\citep{sudhakar2020balancing,ondruska2015scheduled,lahijanian2018resource,mei2005case,brateman2006energy}, which we discuss further in \fref{cp:soa}{Chapter}.


%%%%%%%%%%%%%%%%%%%%%
\section{\color{red}Methodology}



%%%%%%%%%%%%%%%%%%%%%%
\section{Contribution}
\label{sec:contribs}

There are some approaches to merging planning-scheduling in different robotics use cases in the literature~\citep{mei2005case,mei2006deployment,brateman2006energy,zhang2007low,sadrpour2013experimental,sadrpour2013mission,ondruska2015scheduled,lahijanian2018resource,sudhakar2020balancing}, yet the research area remains mostly unexplored~\citep{sudhakar2020balancing,brateman2006energy}. Our work attempts to contribute with the past relevant literature to this existing research gap, proposing coverage planning and scheduling for autonomous aerial robots under stringent energy constraints. Specifically, we derive an energy model for the heterogeneous computing hardware, alongside modeling the energy contribution of the motion. We use the model in an optimal control technique similarly to past literature~\citep{zhang2007low,ondruska2015scheduled,lahijanian2018resource,brateman2006energy}, but fill the gap further with accurate energy modeling of computations and propose an actual implementation of an MPC-based algorithm. We provide a power-saving scheduling policy as opposed to simply varying the frequency of the computing hardware~\citep{zhang2007low,brateman2006energy}, and notably, our approach runs online and is dynamic, planning-scheduling in flight rather than deriving static plans-schedules~\citep{lahijanian2018resource}. It provides additional modeling rigor and incorporates battery-aware optimization as opposed to merely considering the overall energy expenditure~\citep{sudhakar2020balancing}. We vary both the path and the schedule the aerial robot flies and runs, contrary to others that vary just one of the aspects while analyzing the energy implications of the remaining~\citep{ondruska2015scheduled}.

Aside from past relevant studies, literature on topics related to computations energy modeling, battery modeling, motion planning, and aerial planning, our contribution builds from some of our past and forthcoming studies. Our computational energy modeling relies on the methodology in our early study~\citep{seewald2019coarse}, which presented the \powprof{} tool for future energy predictions of heterogeneous computing hardware. We proposed an extension of \powprof{} with a component-based energy modeling approach to abstract per-component energy in a dataflow computational network in another study~\citep{seewald2019component} that we later integrated into a Robot Operating System (ROS)~\citep{quigley2009ros} set-up in the following work~\citep{zamanakos2020energy}. Here we scheduled the simulated tracking of agricultural vehicles. The motion energy model we propose in \fref{cp:model}{Chapter} relies on empirical observations of the Opterra fixed-wing aerial robot flying the agricultural scenario we introduced earlier in our work~\citep{seewald2020mechanical}. We hypothesized the approach on different robots in our brief early study~\citep{seewald2020beyond}. Finally, we plan to detail the planning-scheduling energy-awareness for CPP in precision agriculture in our forthcoming study~\citep{seewald202Xenergy}, relying on all the concepts in this work. 

%%%%%%%%%%%%%%%%%%%
\section{Structure}
\label{sec:structure}

In this chapter, we progressively introduced energy-aware coverage planning and scheduling for autonomous aerial robots, the field of aerial robotics, motivated our study, outlined the approach, and proposed further the applications, methodology, and contributions. We define the problems formally in \fref{cp:pb}{Chapter} and detail the solutions in \fref{cp:dyn}{Chapter}, building upon the remaining chapters. In particular, in \fref{cp:soa}{Chapter}, we present the relevant literature for computations energy and battery modeling, motion and aerial planning, and planning of computations with motion, relating each study to our approach. Planning-scheduling energy awareness requires accurate future energy predictions. We derive the computations and motion energy and battery models in \fref{cp:model}{Chapter}. Here we derive and formally prove a periodic energy model that we use in the remainder. Both coverage planning and energy-aware replanning along scheduling rely on different algorithms we propose in \fref{cp:dyn}{Chapter}. In detail, we derive an optimization algorithm that selects the optimal parameters configuration for the path and computations simultaneously, a CPP algorithm suitable for the use case, and ways to guide the aerial robot physically on the coverage. \fref{cp:res}{Chapter} describes our experimental results, and \fref{cp:conc}{Chapter} concludes with the outcomes and future perspectives. In \fref{app:imp}{Appendix}, we additionally propose the implementations of all the constructs in this work.

%---

