
%
%%%%%%%%%%%%%%%%%%%%%%%%%
%                       %
% Problem Formulation   %
%                       %
%%%%%%%%%%%%%%%%%%%%%%%%%
%
% Brief abstract: formulates the two problem (coverage, (re)planning)
%
% Completion (1-10): 10
% Missing: -
% Proofreading: 11/10/21
% Proofreading: 22/10/21 (integrated reviews)
%
\chapter{Problem Formulation}
\label{cp:pb}

\begin{chapquote}{\cite{arkin2001optimal}}
  ``While we will often speak of the [coverage] problem as `milling' with a `cutter', many of its important applications arise in various contexts outside of machining.''
\end{chapquote}

\vspace*{1em}

\lettrine{T}{he scope of this chapter} is to provide building blocks and formal definitions paving the foundation of the remaining chapters. Here, we derive the planning and coverage problems we solve with this work. The coverage problem is the problem of finding a path, covering all the points in a given space~\citep{choset2001coverage,galceran2013survey}, for instance, the agricultural field in \fref{sec:motivation}{Section}. The coverage path with computations forms the plan. The planning problem is the problem of replanning the plan. In the context of this work, it is replanned energy-wise in the eventuality of energy constraints dissatisfaction and whenever the uncertainty affects the flight unexpectedly. Both the problems are formulated in \fref{sec:2pbs}{Section}. The formulations rely on plan-specific constructs in \fref{sec:definitions}{Sections}\fref{sec:plan}{--\hspace*{-.8ex}}, with the concepts of computations, motion, and computations and motion energies. Further formalities include the difference between computations and motion energies (\Gls{acr:mace}) encountered in \fref{sec:aerial-robo-types}{Section}. We illustrate the problem with an example of the precision agriculture use case in \fref{sec:flight-plan}{Section}.

The chapter connects to the remainder of this work as follows. Here we formalize the plan, the planning and coverage problems, and some other basic constructs. \fref{cp:soa}{Chapter} discusses past approaches to solve the problems. \fref{cp:model}{Chapter} exploits the plan characteristics to derive computations and motion energy and battery models. \fref{cp:dyn}{Chapter} provides a solution to the planning problem with modern optimal control techniques and the coverage problem using a coverage path planning (\Gls{acr:cpp})\findex{coverage path planning} algorithm. It further provides ways to guide the aerial robot with an algorithm based on vector fields, using the plan's building blocks. \fref{cp:res}{Chapter} shows the problems in realistic and simulated experiments.


%%%%%%%%%%%%%%%%%%%%%%%%%%%%%%%%%%%%%%%%%%%%%%%%
\section{Definitions of computations and motion}
\label{sec:definitions}

This section contains definitions of computations and motion, their respective energies, and M\&CE.

\begin{defn}[Computations/motion]
  \label{def:comps}
  \emph{Computations}\findex{computations} are energy-demanding computational tasks. The aerial robot runs the computations on heterogeneous computing hardware that interfaces to microcontrollers.
  
  \emph{Motion}\findex{motion} is the act of the aerial robot moving in the surrounding environment. For this purpose, it runs primitives on microcontrollers that interface to actuators, motors, and other components.
\end{defn}

Autonomous capabilities are often achieved by interconnecting heterogeneous computing hardware and microcontrollers. For the computations, we assume that the heterogeneous computing hardware runs a parametrized schedule. For instance, for the detections in the precision agriculture use case in \fref{sec:motivation}{Section}, a parameter is the frames per second (\Gls{acr:fps})\findex{frames per second} rate. Similarly, for the motion, we assume that the robot travels parametrized paths. In the use case, a parameter changes the coverage quality.

\begin{defn}[Computations/motion and overall energy]
  \label{def:comp-mot-energy}
  Given a path parametrized by $\rho$ values $c_i^\rho:=\{c_{i,1},c_{i,2},\dots,c_{i,\rho}\}$, the \emph{motion energy}\findex{motion energy} is the energy spent by the aerial robot while moving on the path.

  Given a schedule parametrized by $\sigma$ values $c_i^\sigma:=\{c_{i,\rho+1},c_{i,\rho+2},\dots,c_{i,\rho+\sigma}\}$, the \emph{computations energy}\findex{computations energy} is the energy spent by heterogeneous computing hardware executing the schedule.
  
  The \emph{overall energy}\findex{overall energy} is the sum of motion energy and computations energy.
\end{defn}

Physically, motion energy is the energy spent by all the systems powering the aerial robot, excluding the heterogeneous computing hardware. It uses measures in watts for instantaneous or average, joules for overall energies. \fref{sec:defs-stages-triggs}{Section} details the meaning of {\itshape parametrized paths and computations}.


\begin{defn}[Difference of motion and computations energy (\Gls{acr:mace})]\findex{difference of motion and computations energy}
  Given measures of average motion and computations energies, \textit{M\&CE} is the measure of their difference.
\end{defn}

M\&CE is measured in watts and quantifies which of the two energy components is predominant. For \Gls{acr:mace} greater than zero, the motion energy dominates over the computations. The planning-scheduling energy awareness accentuates on an energy-efficient path (e.g., rotary-wing aerial robots \sref{lab:matrice} and \sref{lab:agras} in \fref{fig:robots-vs-power}{Figure}). 
%If we return briefly to \fref{fig:robots-vs-power}{Figure}, it is the case for rotary-wing aerial robots \sref{lab:matrice} and \sref{lab:agras}. 
On the contrary, for \Gls{acr:mace} lower than zero, the computations energy dominates. The planning-scheduling energy awareness focus on a power-saving schedule (e.g., lighter-than-air aerial robot \sref{lab:skye}). 
%It is the case for the lighter-than-air aerial robot \sref{lab:skye} in \fref{fig:robots-vs-power}{Figure}. 
For \Gls{acr:mace} close to zero, both energy components are important energy-wise. The planning-scheduling energy awareness trades both an energy-efficient path and a power-saving schedule to a similar extent (e.g.,  fixed-wing aerial robots \sref{lab:cumulus}\fref{lab:opterra}{--\hspace*{-.8ex}}). %This M\&CE is characteristic of fixed-wing aerial robots, such as \sref{lab:cumulus}\fref{lab:opterra}{--\hspace*{-.8ex}} in \fref{fig:robots-vs-power}{Figure}.


%%%%%%%%%%%%%%%%%%%%%%%%%%%%%%%%%%%%%%
\section{Definition of path functions}
\label{sec:path-functions}

A succession of multiple mathematical functions $\varphi_1,\varphi_2,\dots$ termed path functions forms the coverage path that the aerial robot follows. They are expressed in 2D space at an altitude $h\in\mathbb{R}$ for an inertial navigation frame $\mathcal{O}_W$. Here, we assume the aerial robot flies over a given elemental region for the coverage without altering the altitude\footnote{We discuss later in terms of future directions a generalization of the path functions in 3D space for, e.g., urban monitoring use cases.}.

\begin{defn}[Path functions]
  \label{def:paths}
  $\varphi_i:\mathbb{R}^2\rightarrow\mathbb{R},\,\forall i\in\{1,2,\dots\}$ are \emph{path functions}\findex{path functions}, forming the path. They are a function of a generic time-dependent point $\mathbf{p}(t):=(x_{\mathbf{p}(t)},y_{\mathbf{p}(t)})$ of the aerial robot flying in the 2D space at an altitude $h$ and are continuous and twice differentiable. 
\end{defn}\findex{continuity}\findex{twice differentiability}

\begin{figure}[t!]
  \centering
  
\definecolor{cECECEC}{RGB}{236,236,236}
\definecolor{c989898}{RGB}{152,152,152}
\small

\def \globalscale {1.000000}
\begin{tikzpicture}[y=0.80pt, x=0.80pt, yscale=-.9*\globalscale, xscale=.9*\globalscale, inner sep=0pt, outer sep=0pt]
\path[fill=cECECEC,line join=round,line width=0.160pt] (338.3020,263.8010) -- (381.1050,-0.0001) -- (102.8640,12.1601) -- (58.1440,275.9610) -- (338.3020,263.8010) -- cycle;



\path[draw=c989898,line join=round,line width=0.512pt] (147.6250,213.3720) -- (173.5250,186.3120) -- (132.2780,180.7850);



\path[fill=c989898,line join=round,line width=0.256pt] (163.3290,190.5470) -- (158.3650,192.4980) -- (158.1310,191.9020) -- (163.0950,189.9520) -- (163.3290,190.5470) -- cycle(153.4020,194.4490) -- (148.4380,196.4000) -- (148.2040,195.8040) -- (153.1680,193.8530) -- (153.4020,194.4490) -- cycle(143.4740,198.3510) -- (138.5110,200.3020) -- (138.2760,199.7060) -- (143.2400,197.7550) -- (143.4740,198.3510) -- cycle(133.5470,202.2530) -- (128.5830,204.2040) -- (128.3490,203.6080) -- (133.3130,201.6570) -- (133.5470,202.2530) -- cycle(123.6200,206.1550) -- (118.6560,208.1060) -- (118.4220,207.5100) -- (123.3850,205.5590) -- (123.6200,206.1550) -- cycle(113.6920,210.0560) -- (108.7280,212.0070) -- (108.4940,211.4120) -- (113.4580,209.4610) -- (113.6920,210.0560) -- cycle(103.7650,213.9580) -- (100.0090,215.4350) -- (99.7748,214.8390) -- (103.5310,213.3630) -- (103.7650,213.9580) -- cycle(173.2570,186.6450) -- (168.2930,188.5960) -- (168.0590,188.0010) -- (173.0220,186.0500) -- (173.2570,186.6450) -- cycle;



\path[draw=c989898,line join=round,line width=0.512pt] (55.1846,208.3180) -- (99.4269,215.2590);



\path[draw=black,line join=round,line width=0.512pt] (68.2145,216.7950) -- (99.1080,215.5400);



\path[draw=c989898,line join=round,line width=0.512pt] (99.5793,215.0880) -- (196.7320,112.9950);



\path[draw=foo,line join=round,line width=0.512pt] (139.9040,221.4650) -- (147.6230,213.4870);



\path[draw=black,line join=round,line width=0.512pt] (99.6097,215.3190) -- (377.1250,258.8550);



\path[fill=foo,line join=round,line width=0.256pt] (163.7520,67.1159) -- (158.7880,69.0661) -- (158.5540,68.4704) -- (163.5180,66.5202) -- (163.7520,67.1159) -- cycle(153.8240,71.0164) -- (148.8600,72.9667) -- (148.6260,72.3709) -- (153.5900,70.4207) -- (153.8240,71.0164) -- cycle(143.8960,74.9168) -- (138.9320,76.8670) -- (138.6980,76.2713) -- (143.6620,74.3211) -- (143.8960,74.9168) -- cycle(133.9680,78.8173) -- (129.0040,80.7676) -- (128.7700,80.1719) -- (133.7340,78.2216) -- (133.9680,78.8173) -- cycle(124.0400,82.7178) -- (119.0760,84.6679) -- (118.8420,84.0722) -- (123.8060,82.1220) -- (124.0400,82.7178) -- cycle(114.1120,86.6182) -- (109.1480,88.5685) -- (108.9140,87.9728) -- (113.8780,86.0225) -- (114.1120,86.6182) -- cycle(104.1840,90.5187) -- (99.9651,92.1761) -- (99.7311,91.5804) -- (103.9500,89.9230) -- (104.1840,90.5187) -- cycle(173.6800,63.2155) -- (168.7160,65.1657) -- (168.4820,64.5700) -- (173.4460,62.6198) -- (173.6800,63.2155) -- cycle;



\path[draw=black,line join=round,line width=0.512pt] (99.6290,215.7460) -- (99.6701,45.2765);



\path[fill=black,line join=round,line width=0.160pt] (97.4234,50.0749) -- (99.6963,48.0699) -- (101.7800,50.0657) -- (99.5939,44.2973) -- (97.4234,50.0749) -- cycle;



\path[fill=black,line join=round,line width=0.160pt] (372.7840,255.6340) -- (374.2970,258.2610) -- (371.9260,259.9050) -- (378.0140,258.9110) -- (372.7840,255.6340) -- cycle;



\path[draw=c989898,fill=c989898,line join=round,line width=0.160pt] (191.9960,115.1180) -- (195.0250,115.0410) -- (195.3410,117.9090) -- (197.3700,112.0830) -- (191.9960,115.1180) -- cycle;



\path[draw=black,line join=round,line width=0.512pt] (99.6614,215.4240) -- (348.2110,205.3130);



\path[fill=black,line join=round,line width=0.256pt] (365.8960,204.3560) -- (371.2240,204.1220) -- (371.2520,204.7610) -- (365.9240,204.9950) -- (365.8960,204.3560) -- cycle(376.5520,203.8870) -- (381.8800,203.6530) -- (381.9080,204.2920) -- (376.5800,204.5270) -- (376.5520,203.8870) -- cycle(387.2080,203.4190) -- (391.6130,203.2250) -- (391.6410,203.8640) -- (387.2370,204.0580) -- (387.2080,203.4190) -- cycle(355.2390,204.8250) -- (360.5670,204.5900) -- (360.5960,205.2300) -- (355.2670,205.4640) -- (355.2390,204.8250) -- cycle;



\path[fill=black,line join=round,line width=0.256pt] (51.6916,217.8380) -- (46.3616,218.0290) -- (46.3388,217.3890) -- (51.6687,217.1990) -- (51.6916,217.8380) -- cycle(41.0317,218.2190) -- (35.7017,218.4090) -- (35.6789,217.7700) -- (41.0088,217.5790) -- (41.0317,218.2190) -- cycle(30.3718,218.6000) -- (25.0419,218.7900) -- (25.0190,218.1500) -- (30.3490,217.9600) -- (30.3718,218.6000) -- cycle(62.3514,217.4570) -- (57.0215,217.6480) -- (56.9986,217.0080) -- (62.3286,216.8180) -- (62.3514,217.4570) -- cycle;



\path[draw=black,line join=round,line width=0.512pt] (138.1150,223.3150) -- (140.0590,221.3710);



\path[draw=c989898,line join=round,line width=0.512pt] (129.6630,180.5070) -- (132.3930,180.8300);



\path[draw=c989898,line join=round,line width=0.512pt] (173.5260,186.3500) -- (173.5730,63.0377);



\path[draw=black,line join=round,line width=0.512pt] (96.7492,91.8181) -- (99.4988,91.8249);



\path[cm={{1.0,0.0,0.0,1.0,(376.0,276.0)}}] (0.0000,0.0000) node[above right] () {$x$};



\path[cm={{1.0,0.0,0.0,1.0,(198.0,102.0)}}] (0.0000,0.0000) node[above right] () {$y$};



\path[cm={{1.0,0.0,0.0,1.0,(85.0,46.0)}}] (0.0000,0.0000) node[above right] () {$z$};



\path[draw=black,line join=round,line width=0.512pt] (83.3760,232.1090) -- (99.6032,215.0620);



\path[fill=c989898,line join=round,line width=0.512pt] (173.2360,184.7230) .. controls (174.0900,184.7230) and (174.7820,185.4150) .. (174.7820,186.2700) .. controls (174.7820,187.1240) and (174.0900,187.8160) .. (173.2360,187.8160) .. controls (172.3810,187.8160) and (171.6890,187.1240) .. (171.6890,186.2700) .. controls (171.6890,185.4150) and (172.3810,184.7230) .. (173.2360,184.7230) -- cycle;



\path[cm={{1.0,0.0,0.0,1.0,(161.0,289.0)}}] (0.0000,0.0000) node[above right] () {$\varphi(x,y):=2y-x$};



\path[cm={{1.0,0.0,0.0,1.0,(353.0,224.0)}}] (0.0000,0.0000) node[above right] () {$2y-x=h$};



\path[cm={{1.0,0.0,0.0,1.0,(131.0,238.0)}}] (0.0000,0.0000) node[above right] () {$x_\mathbf{p}$};



\path[cm={{1.0,0.0,0.0,1.0,(113.0,180.9)}}] (0.0000,0.0000) node[above right] () {$y_\mathbf{p}$};



\path[cm={{1.0,0.0,0.0,1.0,(41.0,98.0)}}] (0.0000,0.0000) node[above right] () {$\varphi(\mathbf{p})=d$};



\path[draw=c989898,fill=c989898,line join=round,line width=0.160pt] (175.7670,179.7880) -- (173.4940,181.7930) -- (171.4110,179.7970) -- (173.5970,185.5660) -- (175.7670,179.7880) -- cycle;



\path[draw=c989898,fill=c989898,line join=round,line width=0.160pt] (171.2840,70.1412) -- (173.5570,68.1363) -- (175.6400,70.1321) -- (173.4540,64.3637) -- (171.2840,70.1412) -- cycle;



\path[fill=foo,line join=round,line width=0.512pt] (173.4160,61.7418) .. controls (174.2700,61.7418) and (174.9630,62.4342) .. (174.9630,63.2885) .. controls (174.9630,64.1427) and (174.2700,64.8351) .. (173.4160,64.8351) .. controls (172.5620,64.8351) and (171.8700,64.1427) .. (171.8700,63.2885) .. controls (171.8700,62.4342) and (172.5620,61.7418) .. (173.4160,61.7418) -- cycle;



  \path[fill=cECECEC,line join=round,line width=0.160pt] (182.5940,102.1690) -- (164.2430,102.1690) -- (164.1940,119.9680) -- (182.5870,119.9220) -- (182.5940,102.1690) -- cycle;



  \path[cm={{1.0,0.0,0.0,1.0,(170.0,116.0)}}] (0.0000,0.0000) node[above right] () {$d$};



\path[cm={{1.0,0.0,0.0,1.0,(187.0,186.0)}}] (0.0000,0.0000) node[above right] () {$(x_\mathbf{p},y_\mathbf{p},h)$};


\path[draw=c989898,line join=round,line width=0.512pt] (99.6154,255.8180) -- (99.6251,215.7290);


\end{tikzpicture}


  \caption[Concept of a line as a path function]{The path function is a mathematical function $\varphi(\mathbf{p}(t))=h$ that represents a line at an altitude $h$. A generic point $\mathbf{p}$ in 2D space intersects the plane formed by $\varphi$ at a value $d$ of the $z$-axis.}
  \label{fig:line-pf}
\end{figure}

\begin{figure}[t!]
  \centering
  \input{figures/circle-pf.tikz}
  \caption[Concept of a circle as a path function]{The path function now represents a circle at an altitude $h$. $\mathbf{p}$ intersects the cone formed by $\varphi$ at a value $d_2$ of the $z$-axis.}
  \label{fig:circle-pf}
\end{figure}

We use this notation to guide the aerial robot with the theory of vector fields\findex{vector field}. % in \fref{algo:grad-desc}{Algorithms}\fref{algo:track}{--\hspace*{-.8ex}} in \fref{cp:dyn}{Chapter}. 
For instance, one can define a line as a path function with
\begin{equation}\label{eq:basic-path}
  \varphi(x,y):=ax+by+c,
\end{equation}
where $a,b,c\in\mathbb{R}$ are given constants. The generic point $\mathbf{p}(t)$ intersects $\varphi(x,y)$ at a specific value $d$ of the $z$-axis $(\mathbf{p}(t),d)$. \fref{fig:line-pf}{Figure} illustrates the concept for $c$ zero, $a,b$ minus one and two, and $h$ zero for simplicity. The point intersects the plane\findex{plane} formed by the path function\findex{line}
\begin{equation}\label{eq:pathf-line}
  \varphi(x,y):=2y-x,%-h, there is no minus h here (facepalm)
\end{equation}
at $d=\varphi(\mathbf{p})$. The path that the aerial robot follows is then $\varphi(x,y)=h$. $\varphi(\mathbf{p})-h$ is the distance on the $z$-axis.

Likewise with the line, one can define a circle as a path function with\findex{circle}
\begin{equation}\label{eq:pathf-circle}
  \varphi(x,y):=(x-x_c)^2+(y-y_c)^2-r^2,
\end{equation}
where $x_c,y_c$ are given coordinates of the center and $r\in\mathbb{R}_{>0}$ the radius. \fref{fig:circle-pf}{Figure} illustrates the circle path function for $x_c,y_c$ both three, $\sqrt{r}$ two, and $h$ zero for simplicity. 

In this work, we use lines and circles as path functions. We connect these functions using some specific points--the triggering points--to form the coverage path. However, one can define any mathematical function, with the only requirement being continuity and twice differentiability. The first derivative is a requirement for the vector field, the second derivative of the control action. \fref{cp:dyn}{Chapter} details the concept further.


%%%%%%%%%%%%%%%%%%%%%%%%%%%%%%%%%%%%%%%%%%%%%%%%%%%%%
\section{Definitions of stages and triggering points}
\label{sec:defs-stages-triggs}

The plan has several stages $i=\{1,2,\dots\}$, and we assume that at each stage, the aerial robot runs a schedule and travels a path function $\varphi_i$ using a parameters set $c_i$.
Parameters are variable values to replan the path and computations, influencing the computations/motion energy in \fref{def:comp-mot-energy}{Definition}. The path parameters are real-, the computations parameters integer-valued ($\mathbb{R},\mathbb{Z}$). The notation $c_{i,j}$ denotes the $j$th parameter of the $i$th parameters set
\begin{equation}
  c_i=\{c_{i,1},c_{i,2},\dots,c_{i,j},\dots\}.
\end{equation}

Parameters are bounded. $\underline{c}_{i,j}$ is the lower, $\overline{c}_{i,j}$ the upper bound of the parameter $c_{i,j}$
\begin{equation}
  \underline{c}_{i,j}\leq c_{i,j}\leq\overline{c}_{i,j},
\end{equation}
expressing physical bounds of the computing hardware and the aerial robot (e.g., it is not possible to compute more than the capabilities of the computing hardware, turn narrower than the minimum turning radius of the aerial robot, and so on).

There are $\rho$ \emph{path}-specific \emph{parameters}\findex{path parameters} and $\sigma$ \emph{computations}-specific \emph{parameters}\findex{computations parameters} for every stage. It means that the path at stage $i$ can be replanned with $\rho$ path parameters
$c_i^\rho:=\{c_{i,1},c_{i,2},\dots,c_{i,\rho}\}$, and the computations 
%(i.e., the energy-demanding computational task executed on the computing hardware in \fref{def:comps}{Definition}) 
scheduled with $\sigma$ computations parameters 
$c_i^\sigma:=\{c_{i,\rho+1},c_{i,\rho+2},\dots,c_{i,\rho+\sigma}\}$.

Returning to \fref{sec:definitions}{Section}, ``{\itshape path parametrized by $\rho$ values}'' indicates that $\varphi_i$ is enhanced with $c_i^\rho$. The function $\varphi_i:\mathbb{R}^2\times\mathbb{R}^\rho\rightarrow\mathbb{R}$ is thus a (continuous twice differentiable) function of a point and the path parameters. We use the parameters to alter the path and change the energy consumption. Similarly, ``{\itshape computations parametrized by $\sigma$ values}'' indicates that $c_i^\sigma$ is the computations schedule. These parameters also serve for energy alteration (e.g., decreasing the granularity of a given computation lowers the power). We discuss the alteration of the energy with path and computations parameters further in \fref{sec:nom-cont}{Sections}\fref{sec:merging}{--\hspace*{-.8ex}}.

%Since at every stage the aerial robot travels a path and executes a schedule both parametrized by the parameters, we define 
The stage is a set that contains the path and path and computations parameters.

\begin{defn}[Stage]
  \label{def:stage}
  For a generic point $\mathbf{p}(t)$, the $i$th \emph{stage}\findex{stage} $\Gamma_i$ at time instant $t$ of a plan $\Gamma$ is
  \begin{equation*}\begin{split}
    \Gamma_i:=\{\varphi_i(\mathbf{p}(t),c_i^\rho),c_i^\sigma\mid
    \,&\forall j\,\in\,[\rho]_{>0},\,c_{i,j}\,\,\,\,\,\,\,\in\mathcal{C}_{i,j},\,\\
      &\,\forall k\in[\sigma]_{>0},\,c_{i,\rho+k}\in\mathcal{S}_{i,k}\,\},
  \end{split}\end{equation*}
  where $\mathcal{C}_{i,j}:=[\underline{c}_{i,j},\overline{c}_{i,j}]\subseteq\mathbb{R}$ is the $j$th path parameter constraint set, and $\mathcal{S}_{i,k}:=[\underline{c}_{i,\rho+k},\overline{c}_{i,\rho+k}]\subseteq\mathbb{Z}_{\geq 0}$ the $k$th computation parameter constraint set.\findex{constraint set}
\end{defn}

The next section clarifies why the stage contains the generic point $\mathbf{p}(t)$.

It is possible to merge the computations and path constraint sets in a single constraint set. $i$th stage constraint set is then
\begin{equation}\label{eq:constraint-set}
  \mathcal{U}_i(c_{i,j}):=\begin{cases}
  \mathcal{C}_{i,j} & \text{for } c_{i,j} \text{ with } j\leq\rho\\
  \mathcal{S}_{i,j-\rho} & \text{for } c_{i,j} \text{ with } \rho<j\leq\sigma
\end{cases},\end{equation}
the stage can be simplified 
\begin{equation}
  \Gamma_i:=\{\varphi_i(\mathbf{p}(t),c_i^\rho),c_i^\sigma\mid \,\forall j\,\in\,[\rho+\sigma]_{>0},\,c_{i,j}\in\mathcal{U}_i(c_{i,j})\}.
\end{equation}

Specific points termed triggering points $\mathbf{p}_{\Gamma_i}$ allow the transition between stages.

\begin{defn}[Triggering and final points]\findex{triggering point}\findex{final!point}\findex{final!stage}
  \label{def:trigs}
  The \emph{triggering point} is the point $\mathbf{p}_{\Gamma_{i}}$ that allows the transition between stages. The \emph{final point} is the last triggering point $\mathbf{p}_{\Gamma_{l}}$ relative to the last stage $\Gamma_l$.
\end{defn}

%To move from a given stage $\Gamma_i$ to the next stage $\Gamma_{i+1}$, we define some specific points $\mathbf{p}_{\Gamma_i}$
As soon as the aerial robot reaches the proximity of these points, it switches to the next stage
\begin{equation}\label{eq:trig-vareps}
  \norm{\mathbf{p}(t)-\mathbf{p}_{\Gamma_i}}<\varepsilon_i,
\end{equation}
where $\varepsilon_i\in\mathbb{R}$ is a given stage-dependent constant value expressing the radius of an imaginary circle over the point $\mathbf{p}_{\Gamma_i}$.


%%%%%%%%%%%%%%%%%%%%%%%%%%%%
\section{Definition of plan}
\label{sec:plan}

The energy model in \fref{sec:mot-ener-model}{Section} exploits the concept of the aerial robot flying a set of paths and computations autonomously. Such an autonomous flight plan often presumes a certain degree of periodicity, which one can observe in the precision agriculture use case in \fref{sec:motivation}{Section}. An exhaustive way to cover the agricultural field in \fref{fig:plot2}{Figure}, i.e., to visit all the points in the space, is to define a basic pattern. The aerial robot flies over the field once and iterates the basic pattern until it covers the desired area. In literature, the basic pattern is often termed ``motion'' and the most common is, e.g., boustrophedon motion~\citep{choset2005principles,choset2001coverage,cabreira2019survey,galceran2013survey}. One can then define the plan just as a set of stages, triggering points, a final point, and a shift that specifies how the constructs (except the final point) shifts in space every period. The concept of primitive paths in the following simplifies the planning to this latter case.

\begin{defn}[Primitive paths]
  \label{def:primitive}
  Given $n\in\mathbb{Z}_{>0}$, the paths $\varphi_1,\dots\varphi_n$ are called \emph{primitive paths}\findex{primitive paths} if all the remaining paths in the plan are built from these paths with a \emph{shift}\findex{shift} $\mathbf{d}:=(x_{\mathbf{d}},y_{\mathbf{d}})$. 
\end{defn}

Let us assume the number of stages in the plan is known and is $l\in\mathbb{Z}_{>0}$. If the plan is built from the primitive paths, it means that $n$ in \fref{def:primitive}{Definition} respects
\begin{equation}
  n<l,\,l/n\in\mathbb{Z},
\end{equation}
%This means that $n$ is a multiplier of $l$, whereas $l/n$ is the multiplicand. 
where $n$ is a multiplier of $l$. 

One can then write the remaining paths from the $n$ primitive paths as $\varphi_{n+1},$ $\varphi_{n+2},\dots,\varphi_{n+n},\dots,\varphi_l$, or more generally $\varphi_{(i-1)n+1},$ $\varphi_{(i-1)n+2},\dots,\varphi_{(i-1)n+n},\,\forall i\in[l/n-1]_{>0}$. Or similarly
\begin{equation}\label{eq:primitive}\begin{split}
  &\varphi_{(i-1)n+j}(\mathbf{p}+(i-1)\mathbf{d},c_1^\rho)-\varphi_{in+j}(\mathbf{p}+i\mathbf{d},c_1^\rho)=e_j,
\end{split}\end{equation}
for a given shift $\mathbf{d}$, initial point $\mathbf{p}$, and initial value of path parameters $c_1^\rho$. \frefeq{eq:primitive} holds $\forall i\in[l/n-1]_{>0},j\in[n]_{>0}$. $e_j\in\mathbb{R}$ is the $j$th constant difference.

Generally, if the plan is built from the primitive paths, it is not required to know a priori the number of stages $l$. The paths can be iterated up until the final point $\mathbf{p}_{\Gamma_l}$. 
%We show in \fref{cp:model}{Chapter} from collected energy data that the plans built from primitive paths often have a periodic energy evolution. 
Alternatively to the primitive paths, one can define the plan as a mere linear succession of stages along with the triggering and final points. In the latter case, the energy can be periodic, aperiodic, or semi-periodic. Aperiodicity affects the modeling, and thus future energy predictions. We discuss the concrete meaning of aperiodicity and semi-periodicity in the context of energy modeling in \fref{sec:non-perio}{Section}.

Formally, the plan is a finite state machine that exploits the constructs of path functions in \fref{def:paths}{Definition} and stages and triggering points in \fref{def:stage}{Definitions}\fref{def:trigs}{--\hspace*{-.8ex}}.

\begin{defn}[Plan]
  \label{def:plan}
  For a generic point $\mathbf{p}(t)$, the \emph{plan} is a finite state machine\findex{finite state machine} (\Gls{acr:fsm}) $\Gamma$, where the state-transition function $s:\bigcup_i{\Gamma_i}\times\mathbb{R}^2\rightarrow\bigcup_i{\Gamma_i}$ maps a stage and a point to the next stage
  \begin{equation*}s(\Gamma_i,\mathbf{p}(t)):=\begin{cases}
    \Gamma_{i+j} & \exists j\in\mathbb{Z},\text{ if }\norm{\mathbf{p}(t)-\mathbf{p}_{\Gamma_i}}<\varepsilon_i\\
    \Gamma_i & \text{otherwise}
  \end{cases}.\end{equation*}
\end{defn}

The value $\varepsilon_i$ in \fref{def:plan}{Definition} is the same $\varepsilon_i$ in \frefeq{eq:trig-vareps}. 

A concept that we use in the remainder of this work, and particularly in energy modeling in \fref{cp:model}{Chapter}, is the period--the time required to fly the primitive paths $\varphi_1,\varphi_2,\dots,\varphi_n$ (or generally $\varphi_{(i-1)n+1},$ $\varphi_{(i-1)n+2},\dots,\varphi_{(i-1)n+n}$). 

\begin{defn}[Period]
  \label{def:period}\findex{period}
  For a given stage $\Gamma_i$ and $j\in\mathbb{Z}_{>0}$, the \emph{period}\findex{period} $T\in\mathbb{R}_{> 0}$ is the flight time measured in seconds between $\varphi_{(i-1)n+j}$ and $\varphi_{in+j}$.
\end{defn} 
  
We assume the initial period $T$ is one and measure the period required to fly the paths physically or in simulation. The periods might be different for different $j$s due to atmospheric interferences or replanning. For the path functions, the coverage algorithm defines the plan using primitive paths $\varphi_1,\varphi_2,\dots,\varphi_n$, but can alternatively define all the stages explicitly and find $n$ searching the value which satisfies the \frefeq{eq:primitive}. If there is no such value (e.g., when the plan is composed of only one stage or the plan is aperiodic), the period can be determined empirically from energy data.

We illustrate the plan, stage, triggering, and final points definitions in \fref{fig:state-machine}{Figures}\fref{fig:state-machine-loop}{--\hspace*{-.8ex}}.

\begin{figure}[h!]
  \center
  \begin{tikzpicture}[shorten >=.5pt,node distance=12.5ex,on grid,auto,initial text=\footnotesize\fontfamily{phv}\selectfont{start}]
    \node[state,initial] (q_i) {$\Gamma_1$}; 
    \node        [right=of q_i] (q_dots0) {$\cdots$};
    \node[state] (q_0) [right=of q_dots0] {$\Gamma_i$};
    \node        (q_dots1) [right=of q_0] {$\cdots$};
    \node[state,accepting] (q_f) [right=of q_dots1] {$\Gamma_f$};
    \path[->]
    (q_i) edge node {$\mathbf{p}_{\Gamma_{1}}$} (q_dots0)
    (q_dots0) edge node{$\mathbf{p}_{\Gamma_{i-1}}$} (q_0)
    (q_0) edge node {$\mathbf{p}_{\Gamma_i}$} (q_dots1)
    (q_dots1) edge node {$\mathbf{p}_{\Gamma_{l}}$} (q_f)    
    (q_i) edge [loop above] node {$\mathbf{p}(t_1)$} (q_i)
    (q_0) edge [loop above] node {$\mathbf{p}(t_2)$} (q_0)
    (q_f) edge [loop above] node {$\mathbf{p}(t_3)$} (q_f)
    ; %end path 
    \draw [decorate,decoration={brace,amplitude=10pt,mirror,raise=10pt},yshift=0pt]
    (q_i.south west) -- (q_f.south west) node [black,midway,yshift=-9ex]{$\Gamma$};
  \end{tikzpicture}
  \caption[Definition of a plan]{Definition of a plan $\Gamma$ as an FSM. Each state is a stage $\Gamma_i$, the transition happens in the proximity of specific points called triggering points $\mathbf{p}_{\Gamma_i}$. The accepting stage $\Gamma_f$ indicates the termination of the plan.}
  \label{fig:state-machine}
\end{figure}
\fref{fig:state-machine}{Figure} illustrates a plan with a linear succession of stages. The triggering point $\mathbf{p}_{\Gamma_{i-1}}$ allows the transition to the stage $\Gamma_i$. The robot remains in the stage with any generic point $\mathbf{p}(t_2)$, where $t_1<t_2<t_3$ are three different time instants. It eventually enters the stage $\Gamma_{i+1}$ with the triggering point $\mathbf{p}_{\Gamma_i}$ and so on, until it reaches the final point. $\Gamma_f$ is the accepting stage (it indicates that the robot has completed the plan).
\begin{figure}[h!]
  \center
  \begin{tikzpicture}[shorten >=1pt,node distance=27ex,on grid,auto]
    \node        (q_dots0) {$\cdots$};
    \node[state] (q_0) [right=of q_dots0] {$\Gamma_i$};
    \node        (q_dots1) [right=of q_0] {$\cdots$};   
    \path[->]
    (q_dots0) edge node{$\mathbf{p}_{\Gamma_{i-1}}(c_1^\rho,\dots,c_{i-1}^\rho)$} (q_0)
    (q_0) edge node {$\mathbf{p}_{\Gamma_i}(c_1^\rho,\dots,c_{i}^\rho)$} (q_dots1)    
    (q_0) edge [loop above] node {$\mathbf{p}(t_2)$} (q_0)
    ; %end path
  \end{tikzpicture}
  \caption[Detail of a stage in the FSM]{Detail of the stage $\Gamma_i$ in the FSM. The triggering points used to transition between states are expressed in the function of the last and/or previous path parameters.}
  \label{fig:state-machine2}
\end{figure}
\fref{fig:state-machine2}{Figure} illustrates that it is possible to express the basic constructs--such as path functions and triggering points--in the function of the $i$th path parameters $c_{i}^{\rho}$, or any previous path parameters, propagating the information therein if necessary. We further expand on this notion in the example in \fref{sec:flight-plan}{Section}, where we propagate a path parameter to all the following triggering points and path functions.

\begin{figure}[h!]
  \center
  \begin{tikzpicture}[shorten >=.5pt,node distance=12.5ex,on grid,auto,initial text=\footnotesize\fontfamily{phv}\selectfont{start}]
    \node[state,initial] (q_i) {$\Gamma_1$}; 
    \node[state] (q_2) [right=of q_i] {$\Gamma_2$}; 
    \node        [right=of q_2] (q_dots0) {$\cdots$};
    \node[state] (q_0) [right=of q_dots0] {$\Gamma_n$};
    \node[state,accepting] (q_f) [right=of q_0] {$\Gamma_f$};
    \path[->]
    (q_i) edge node {$\mathbf{p}_{\Gamma_{1}}$} (q_2)
    (q_2) edge node {$\mathbf{p}_{\Gamma_{2}}$} (q_dots0)
    (q_dots0) edge node{$\mathbf{p}_{\Gamma_{n-1}}$} (q_0)
    (q_0) edge [bend right=65] node [above] {$\mathbf{p}_{\Gamma_n}$} (q_i)
    (q_i) edge [bend left=-65] node [above] {$\mathbf{p}_{\Gamma_l}$} (q_f)
    (q_2) edge [bend left=-45] node [above] {$\mathbf{p}_{\Gamma_l}$} (q_f)
    (q_0) edge node {$\mathbf{p}_{\Gamma_{l}}$} (q_f)    
    (q_i) edge [loop above] node {$\mathbf{p}(t_1)$} (q_i)
    (q_2) edge [loop above] node {$\mathbf{p}(t_2)$} (q_2)
    (q_0) edge [loop above] node {$\mathbf{p}(t_3)$} (q_0)
    (q_f) edge [loop above] node {$\mathbf{p}(t_4)$} (q_f)
    ; %end path 
    \draw [decorate,decoration={brace,amplitude=10pt,mirror,raise=60pt},yshift=0pt]
    (q_i.south west) -- (q_f.south west) node [black,midway,yshift=-22ex]{$\Gamma$};
  \end{tikzpicture}
  \caption[Definition of a plan with a loop]{Definition of a plan $\Gamma$ with periodic patterns. Stages $\Gamma_1,\Gamma_2,\dots,\Gamma_n$ containing primitive paths $\varphi_1,\varphi_2,\dots,\varphi_n$ are iterated with a shift $\mathbf{d}$.}
  \label{fig:state-machine-loop}
\end{figure}
\fref{fig:state-machine-loop}{Figure} illustrates a plan composed of $n$ stages $\Gamma_1,\Gamma_2,\dots,\Gamma_n$ (containing primitive paths $\varphi_1,\varphi_2,\dots,\varphi_n$) that are reiterated with the shift $\mathbf{d}$. $t_3<t_4$ is another time instant. We will see in \fref{cp:dyn}{Chapter} that the algorithm that solves the coverage problem outputs primitive paths, and the corresponding plan to be replanned is equivalent to \fref{fig:state-machine-loop}{Figure}. 


%%%%%%%%%%%%%%%%%%%%%%%%%%%
\section{Problem Statement}
\label{sec:2pbs}

The overall strategy for energy-aware coverage planning and scheduling for autonomous aerial robots is split into two sub-problems: the planning and coverage problems.  

\subsection{Planning problem}
\label{sec:plan-pb}

The planning problem is the problem of finding the optimal configurations of the parameters using some criteria. Within planning-scheduling energy awareness in the robotics research literature, it is often solved with optimal control techniques~\citep{zhang2007low,ondruska2015scheduled,lahijanian2018resource,brateman2006energy}. Here, we focus on energy criteria, such as the battery constraints. %In particular, in the solution to the planning problem in \fref{sec:mpc}{Section} uses a cost function (i.e., the function to maximize) that incorporates the energy model in \fref{sec:mot-ener-model}{Section}. 
Solving the planning problem by finding the optimal configurations with different criteria, such as shortest time, highest security, path tracking with the shortest detour, or others, is equally possible. In the context of the TeamPlay project that funded a considerable part of this work, we aim to find, for instance, the tradeoffs between time, energy, and security criteria for a variety of use cases. \fref{cp:conc}{Chapter} discusses the eventuality of solving the planning problem with different costs. 
%We now use the concepts in the previous sections and provide a formal definition of the planning problem that we are interested in solving in the remainder of this work.

\begin{pb}[Planning problem]
  \label{pb}\findex{planning problem}
  Consider an initial plan $\Gamma$ in \fref{def:plan}{Definition}. It is either composed of $l$ stages or $n$ stages and a shift $\mathbf{d}$. The \emph{planning problem} is the problem of finding the optimal configuration of \emph{path} and \emph{computations parameters} $c_i\,\forall i\in[l]_{>0}$ or $\forall i\in\{1,2,\dots\}$ under energy constraints and uncertainty at each time step $t$.
\end{pb}    

In the definition, there is a fixed or variable number of stages, in the sense that all the stages are reiterated using the primitive paths in \fref{def:primitive}{Definition} up until the aerial robot reaches the last triggering point $\mathbf{p}_l$. Given the optimal configuration, we are further interested in the guidance to the path and the scheduling of the computations.

\subsection{Coverage problem}

%Given a polygon representing the space to be covered, some possible obstacles within the polygon, a starting point, and a turning radius, 
The coverage problem is the problem of finding a route that covers all the points in a given space, avoiding the obstacles. In the literature, it is under a branch of motion planning termed CPP~\citep{choset1998coverage,choset2001coverage,galceran2013survey}. Some approaches ensure the completeness of the coverage and include algorithms optimized for mobile robots. Here, we focus on these algorithms, including specific constraints such as turning radius and the variability of the coverage.
%We discuss the state-of-the-art in CPP in \fref{cp:soa}{Chapter}. 
Furthermore, in the aerial robotics context, obstacles\findex{obstacles} can be seen as areas that the aerial robot should not consider~\citep{cabreira2019survey}. Since there are no obstacles in our use case to avoid physically--e.g., airports, tall buildings, etc.--we use the term no-interest zones (\Gls{acr:niz}s)\findex{no-interest zones}. It depicts areas on the ground where the aerial robot can potentially fly but should not perform any computation, e.g., adjacent agricultural lots or supporting infrastructure laying on the ground. We assume that sets of vertices describe the space and NIZs. In this context, the coverage problem is the problem of covering the space while, e.g., overflying NIZs without computing. 
\fref{cp:conc}{Chapter} discusses a generalization of NIZs to physical obstacles. 
%We discuss this methodology further in \fref{sec:cov-path-plan}{Section}.

\begin{pb}[Coverage problem]
  \label{pb:cov-pb}\findex{coverage problem}\findex{verices}
  Consider a finite set of vertices of a polygon to be covered $v:=\{v_1,v_2,\dots\}$ and of NIZs $o:=\{o_1:=\{o_{1,1},o_{1,2},\dots\},o_2:=\{o_{2,1},o_{2,2},\dots\},\dots\}$ where each vertex $v_i:=(x_{v_i},y_{v_i}),\forall i\in|v|,\,o_{j,k}:=(x_{o_{j,k}},y_{o_{j,k}}),\,\forall j\in|o|,k\in|o_j|$ is a point w.r.t. $\mathcal{O}_W$. Let $\underline{r}\in\mathbb{R}_{\geq 0}$, the minimum turning radius, and $\mathbf{p}(t_0)$, the starting point at the time instant $t_0$, be given. The \emph{coverage problem} is the problem of finding a plan $\Gamma$ that covers $v$ while avoiding computations over $o$\footnote{There might be a further strict requirement to consider obstacles as no-flight zones (\Gls{acr:nfz}s) rather than NIZs, which we discuss in the context of possible future directions.}, starting from $\mathbf{p}(t_0)$ with a turning radius greater or equal to $\underline{r}\in\mathbb{R}_{\geq 0}$.
\end{pb}    

\fref{cp:dyn}{Chapter} proposes two algorithms for the problems in this section. A first algorithm generates the coverage path, and another algorithm replans the plan in time. The replanning is energy-aware. The algorithm outputs the best trajectory of the path and computations alterations for an aerial robot with varying battery and atmospheric conditions. 


%%%%%%%%%%%%%%%%%%%%%%%%%%%%%%%%%%%%%%%
\section{Precision agriculture use case}
\label{sec:flight-plan}

In this section, we discuss an example of a plan for the Opterra fixed-wing aerial robot\findex{Opterra fixed-wing aerial robot} in the precision agriculture use case. Here, we provide one possible solution to \fref{pb:cov-pb}{Problem}, which we will generalize in \fref{cp:dyn}{Chapter}. Path-wise, the aerial robot covers a polygon with variable quality of coverage. Computation-wise, it detects ground hazards and communicates eventual detection with other ground-based actors. To simplify the notation, we split the plan into two sub-plans, one containing the paths and the other the computations. We discuss the first sub-plan in \fref{sec:path-wise}{Section} and the second in \fref{sec:computation-wise}{Section}.

\subsection{Paths sub-plan}
\label{sec:path-wise}

The paths sub-plan defines the paths that form the plan. For simplicity, we assume the polygon has four sides with four vertices $v:=\{v_1,\dots,v_4\}$ (it is a rectangle) and has no NIZs $o:=\emptyset$. An intuitive static plan $\Gamma$ can be composed of lines connected by circles. The distance between the lines can then be a measure of the quality of the coverage. Such an intuitive static plan is illustrated in \fref{fig:bm-like_pb}{Figure}.

\begin{figure}[ht!]
  \centering
  \input{figures/bm-like_pb.tikz}
  \caption[Intuitive plan to cover a regular polygon with four sides]{An intuitive plan to cover a regular polygon with four sides composed of circles $\varphi_2,\varphi_4$ and lines $\varphi_1,\varphi_3$. To switch between the paths, the aerial robot reaches the proximity of triggering points $\mathbf{p}_{\Gamma_1}:=(x_{\Gamma_1},y_{\Gamma_1}),\mathbf{p}_{\Gamma_2},\mathbf{p}_{\Gamma_3},\mathbf{p}_{\Gamma_4}$. The dashed blue line indicates the triggering points, and the black line is the planned flight. The rest of the polygon is covered by iterating the primitive paths and triggering points with a shift.}
  \label{fig:bm-like_pb}
\end{figure}

We can use the intuitive plan to solve the coverage problem with a rotary-wing aerial robot with a small turning radius. A similar pattern, termed the boustrophedon motion\findex{boustrophedon motion}, is abundant in the aerial robotics literature relative to CPP~\citep{difranco2015energy,araujo2013multiple,artemenko2016energy,cabreira2018energy} and in the broader mobile robotics literature~\citep{choset2005principles,lavalle2006planning,choset2001coverage} where the circles are instead straight lines parallel to the segments connecting the corresponding vertices. However, this pattern is unsuitable for a fixed-wing aerial robot such as the Opterra. Fixed-wing aerial robots have reduced maneuverability compared to rotary-wings~\citep{dille2013efficient,mannadiar2010optimal,xu2011optimal,xu2014efficient} and are generally unable to fly quick turns~\citep{wang2017curvature}.

We illustrate an updated version of the intuitive plan for fixed-wing aerial robots in \fref{fig:zambo-like_pb}{Figure}. This latter coverage variation is similar to Zamboni motion\findex{Zamboni motion} in the literature~\citep{araujo2013multiple}. 
We term the plans in \fref{fig:bm-like_pb}{Figure} and \fref{fig:zambo-like_pb}{Figure} boustrophedon-like and Zamboni-like motions\findex{boustrophedon-like motion}\findex{Zamboni-like motion} because they are similar to the robotics literature but optimized to our use case and potentially a broad class of aerial robots with turning constraints. Indeed such constraints are commonly treated in the aerial robotics literature~\citep{artemenko2016energy,li2011coverage,xu2011optimal,xu2014efficient}, and similar patterns are already in other studies~\citep{huang2001optimal,xu2014efficient}. We discuss further the boustrophedon, Zamboni, and other motions for the coverage in \fref{cp:soa}{Chapter}. 

The Zamboni-like motion in \fref{fig:zambo-like_pb}{Figure} is composed of four primitive paths
\begin{subequations}\label{eq:basic-plan}\begin{align}
\varphi_1(\mathbf{p}(t))&:=x-10,\label{eq:line1}\\
\varphi_2(\mathbf{p}(t))&:=(x-85)^2+(y-10)^2-5625,\label{eq:circ1}\\
\varphi_3(\mathbf{p}(t))&:=x-160,\label{eq:line2}\\
\varphi_4(\mathbf{p}(t))&:=(x-90)^2+(y-140)^2-4900,\label{eq:circ2}\end{align}
\end{subequations}
where $x,y$s are the $x$- and $y$-coordinates of a generic point $\mathbf{p}(t)$. The triggering points (the points in which proximity occurs the change of stages) are then the points
\begin{equation}\label{eq:basic-plan-trigs}
  \mathbf{p}_{\Gamma_1}:=(10,10),\,\mathbf{p}_{\Gamma_2}:=(160,10),\,\mathbf{p}_{\Gamma_3}:=(160,140),\,\mathbf{p}_{\Gamma_4}:=(20,140).
\end{equation}

\begin{figure}[ht!]
  \centering
  
\definecolor{c989898}{RGB}{152,152,152}
\small

\def \globalscale {.670000}
\begin{tikzpicture}[y=0.80pt, x=0.80pt, yscale=-\globalscale, xscale=\globalscale, inner sep=0pt, outer sep=0pt]
\path[fill=foo,line join=round,line width=0.256pt] (31.2248,313.3740) -- (36.5581,313.3740) -- (36.5581,314.0140) -- (31.2248,314.0140) -- (31.2248,313.3740) -- cycle(41.8915,313.3740) -- (47.2248,313.3740) -- (47.2248,314.0140) -- (41.8915,314.0140) -- (41.8915,313.3740) -- cycle(52.5581,313.3740) -- (57.8915,313.3740) -- (57.8915,314.0140) -- (52.5581,314.0140) -- (52.5581,313.3740) -- cycle(63.2248,313.3740) -- (68.5581,313.3750) -- (68.5581,314.0150) -- (63.2248,314.0140) -- (63.2248,313.3740) -- cycle(73.8915,313.3750) -- (79.2248,313.3750) -- (79.2248,314.0150) -- (73.8915,314.0150) -- (73.8915,313.3750) -- cycle(84.5581,313.3750) -- (89.8915,313.3750) -- (89.8915,314.0150) -- (84.5581,314.0150) -- (84.5581,313.3750) -- cycle(95.2248,313.3750) -- (100.5580,313.3750) -- (100.5580,314.0150) -- (95.2248,314.0150) -- (95.2248,313.3750) -- cycle(105.8910,313.3750) -- (111.2250,313.3750) -- (111.2250,314.0150) -- (105.8910,314.0150) -- (105.8910,313.3750) -- cycle(116.5580,313.3750) -- (121.8910,313.3750) -- (121.8910,314.0150) -- (116.5580,314.0150) -- (116.5580,313.3750) -- cycle(127.2250,313.3760) -- (132.5580,313.3760) -- (132.5580,314.0160) -- (127.2250,314.0160) -- (127.2250,313.3760) -- cycle(137.8910,313.3760) -- (143.2250,313.3760) -- (143.2250,314.0160) -- (137.8910,314.0160) -- (137.8910,313.3760) -- cycle(148.5580,313.3760) -- (153.8910,313.3760) -- (153.8910,314.0160) -- (148.5580,314.0160) -- (148.5580,313.3760) -- cycle(159.2250,313.3760) -- (164.5580,313.3760) -- (164.5580,314.0160) -- (159.2250,314.0160) -- (159.2250,313.3760) -- cycle(169.8910,313.3760) -- (175.2250,313.3760) -- (175.2250,314.0160) -- (169.8910,314.0160) -- (169.8910,313.3760) -- cycle(180.5580,313.3760) -- (185.8910,313.3770) -- (185.8910,314.0170) -- (180.5580,314.0160) -- (180.5580,313.3760) -- cycle(191.2250,313.3770) -- (196.5580,313.3770) -- (196.5580,314.0170) -- (191.2250,314.0170) -- (191.2250,313.3770) -- cycle(201.8910,313.3770) -- (207.2250,313.3770) -- (207.2250,314.0170) -- (201.8910,314.0170) -- (201.8910,313.3770) -- cycle(212.5580,313.3770) -- (217.8910,313.3770) -- (217.8910,314.0170) -- (212.5580,314.0170) -- (212.5580,313.3770) -- cycle(223.2250,313.3770) -- (228.5580,313.3770) -- (228.5580,314.0170) -- (223.2250,314.0170) -- (223.2250,313.3770) -- cycle(233.8910,313.3770) -- (239.2250,313.3780) -- (239.2250,314.0180) -- (233.8910,314.0170) -- (233.8910,313.3770) -- cycle(244.5580,313.3780) -- (249.8910,313.3780) -- (249.8910,314.0180) -- (244.5580,314.0180) -- (244.5580,313.3780) -- cycle(20.5581,313.3740) -- (25.8915,313.3740) -- (25.8915,314.0140) -- (20.5581,314.0140) -- (20.5581,313.3740) -- cycle;



\path[fill=black,line join=round,line width=0.160pt] (17.0526,14.2224) -- (20.3255,12.2175) -- (23.4090,14.2133) -- (20.2230,6.4450) -- (17.0526,14.2224) -- cycle;



\path[fill=black,line join=round,line width=0.160pt] (300.9970,324.7320) -- (302.9740,328.0300) -- (300.9520,331.0880) -- (308.7470,327.9750) -- (300.9970,324.7320) -- cycle;



\path[draw=c989898,line join=round,line width=0.256pt] (252.0790,17.3728) -- (252.0790,429.9190);



\path[draw=c989898,line join=round,line width=0.256pt] (34.8827,17.3632) -- (34.8827,429.9090);



\path[draw=c989898,line join=round,line width=0.256pt] (150.8380,125.0660) ellipse (2.8574cm and 2.8574cm);



\path[cm={{1.0,0.0,0.0,1.0,(75.0,40.0)}}] (0.0000,0.0000) node[above right] () {$\varphi_4$};


\path[fill=foo,line join=round,line width=0.256pt] (49.2470,317.0060) -- (49.2470,311.6730) -- (49.8870,311.6730) -- (49.8870,317.0060) -- (49.2470,317.0060) -- cycle(49.2470,306.3400) -- (49.2470,301.0060) -- (49.8870,301.0060) -- (49.8870,306.3400) -- (49.2470,306.3400) -- cycle(49.2470,295.6730) -- (49.2470,290.3400) -- (49.8870,290.3400) -- (49.8870,295.6730) -- (49.2470,295.6730) -- cycle(49.2470,285.0060) -- (49.2470,279.6730) -- (49.8870,279.6730) -- (49.8870,285.0060) -- (49.2470,285.0060) -- cycle(49.2470,274.3400) -- (49.2470,269.0060) -- (49.8870,269.0060) -- (49.8870,274.3400) -- (49.2470,274.3400) -- cycle(49.2470,263.6730) -- (49.2470,258.3400) -- (49.8870,258.3400) -- (49.8870,263.6730) -- (49.2470,263.6730) -- cycle(49.2470,253.0060) -- (49.2470,247.6730) -- (49.8870,247.6730) -- (49.8870,253.0060) -- (49.2470,253.0060) -- cycle(49.2470,242.3400) -- (49.2470,237.0060) -- (49.8870,237.0060) -- (49.8870,242.3400) -- (49.2470,242.3400) -- cycle(49.2470,231.6730) -- (49.2470,226.3400) -- (49.8870,226.3400) -- (49.8870,231.6730) -- (49.2470,231.6730) -- cycle(49.2470,221.0060) -- (49.2470,215.6730) -- (49.8870,215.6730) -- (49.8870,221.0060) -- (49.2470,221.0060) -- cycle(49.2470,210.3400) -- (49.2470,205.0060) -- (49.8870,205.0060) -- (49.8870,210.3400) -- (49.2470,210.3400) -- cycle(49.2470,199.6730) -- (49.2470,194.3400) -- (49.8870,194.3400) -- (49.8870,199.6730) -- (49.2470,199.6730) -- cycle(49.2470,189.0060) -- (49.2470,183.6730) -- (49.8870,183.6730) -- (49.8870,189.0060) -- (49.2470,189.0060) -- cycle(49.2470,178.3400) -- (49.2470,173.0060) -- (49.8870,173.0060) -- (49.8870,178.3400) -- (49.2470,178.3400) -- cycle(49.2470,167.6730) -- (49.2470,162.3400) -- (49.8870,162.3400) -- (49.8870,167.6730) -- (49.2470,167.6730) -- cycle(49.2470,157.0060) -- (49.2470,151.6730) -- (49.8870,151.6730) -- (49.8870,157.0060) -- (49.2470,157.0060) -- cycle(49.2470,146.3400) -- (49.2470,141.0060) -- (49.8870,141.0060) -- (49.8870,146.3400) -- (49.2470,146.3400) -- cycle(49.2470,135.6730) -- (49.2470,130.3400) -- (49.8870,130.3400) -- (49.8870,135.6730) -- (49.2470,135.6730) -- cycle(49.2470,125.0060) -- (49.2470,123.4470) -- (49.8870,123.4470) -- (49.8870,125.0060) -- (49.2470,125.0060) -- cycle(49.2470,327.6730) -- (49.2470,322.3400) -- (49.8870,322.3400) -- (49.8870,327.6730) -- (49.2470,327.6730) -- cycle;



\path[cm={{1.0,0.0,0.0,1.0,(24.0,14.0)}}] (0.0000,0.0000) node[above right] () {$\varphi_1$};



\path[draw=black,line join=round,line width=0.512pt] (10.0117,328.0060) -- (278.0320,328.0060) -- (281.8190,334.3940) -- (286.9390,323.1940) -- (290.4590,328.0060) -- (305.2120,328.0060);



\path[draw=black,line join=round,line width=0.512pt] (20.3262,438.0360) -- (20.3268,10.8361);



\path[cm={{1.0,0.0,0.0,1.0,(320.0,341.0)}}] (0.0000,0.0000) node[above right] () {$x$};

\path[cm={{1.0,0.0,0.0,1.0,(5.0,6.0)}}] (0.0000,0.0000) node[above right] () {$y$};



\path[cm={{1.0,0.0,0.0,1.0,(-6.0,128.0)}}] (0.0000,0.0000) node[above right] () {$y_{\Gamma_3}$};



\path[cm={{1.0,0.0,0.0,1.0,(-5.0,318.0)}}] (0.0000,0.0000) node[above right] () {$y_{\Gamma_1}$};



\path[cm={{1.0,0.0,0.0,1.0,(24.0,345.0)}}] (0.0000,0.0000) node[above right] () {$x_{\Gamma_1}$};



\path[cm={{1.0,0.0,0.0,1.0,(51.0,345.0)}}] (0.0000,0.0000) node[above right] () {$x_{\Gamma_4}$};



\path[cm={{1.0,0.0,0.0,1.0,(256.0,345.0)}}] (0.0000,0.0000) node[above right] () {$x_{\Gamma_2}$};




\path[draw=c989898,line join=round,line width=0.256pt] (143.4670,313.1890) ellipse (3.0633cm and 3.0633cm);



\path[draw=black,line join=round,line width=0.512pt] (34.8827,123.8030) -- (34.8827,313.6480);



\path[fill=foo,line join=round,line width=0.256pt] (31.2249,123.2740) -- (36.5582,123.2740) -- (36.5582,123.9140) -- (31.2249,123.9140) -- (31.2249,123.2740) -- cycle(41.8915,123.2740) -- (47.2249,123.2740) -- (47.2249,123.9140) -- (41.8915,123.9140) -- (41.8915,123.2740) -- cycle(52.5582,123.2740) -- (57.8915,123.2740) -- (57.8915,123.9140) -- (52.5582,123.9140) -- (52.5582,123.2740) -- cycle(63.2249,123.2740) -- (68.5582,123.2740) -- (68.5582,123.9140) -- (63.2249,123.9140) -- (63.2249,123.2740) -- cycle(73.8915,123.2740) -- (79.2249,123.2750) -- (79.2249,123.9150) -- (73.8915,123.9150) -- (73.8915,123.2740) -- cycle(84.5582,123.2750) -- (89.8915,123.2750) -- (89.8915,123.9150) -- (84.5582,123.9150) -- (84.5582,123.2750) -- cycle(95.2249,123.2750) -- (100.5580,123.2750) -- (100.5580,123.9150) -- (95.2249,123.9150) -- (95.2249,123.2750) -- cycle(105.8920,123.2750) -- (111.2250,123.2750) -- (111.2250,123.9150) -- (105.8920,123.9150) -- (105.8920,123.2750) -- cycle(116.5580,123.2750) -- (121.8920,123.2750) -- (121.8920,123.9150) -- (116.5580,123.9150) -- (116.5580,123.2750) -- cycle(127.2250,123.2750) -- (132.5580,123.2750) -- (132.5580,123.9150) -- (127.2250,123.9150) -- (127.2250,123.2750) -- cycle(137.8920,123.2760) -- (143.2250,123.2760) -- (143.2250,123.9160) -- (137.8920,123.9150) -- (137.8920,123.2760) -- cycle(148.5580,123.2760) -- (153.8920,123.2760) -- (153.8920,123.9160) -- (148.5580,123.9160) -- (148.5580,123.2760) -- cycle(159.2250,123.2760) -- (164.5580,123.2760) -- (164.5580,123.9160) -- (159.2250,123.9160) -- (159.2250,123.2760) -- cycle(169.8920,123.2760) -- (175.2250,123.2760) -- (175.2250,123.9160) -- (169.8920,123.9160) -- (169.8920,123.2760) -- cycle(180.5580,123.2760) -- (185.8920,123.2760) -- (185.8920,123.9160) -- (180.5580,123.9160) -- (180.5580,123.2760) -- cycle(191.2250,123.2760) -- (196.5580,123.2760) -- (196.5580,123.9160) -- (191.2250,123.9160) -- (191.2250,123.2760) -- cycle(201.8920,123.2770) -- (207.2250,123.2770) -- (207.2250,123.9170) -- (201.8920,123.9170) -- (201.8920,123.2770) -- cycle(212.5580,123.2770) -- (217.8920,123.2770) -- (217.8920,123.9170) -- (212.5580,123.9170) -- (212.5580,123.2770) -- cycle(223.2250,123.2770) -- (228.5580,123.2770) -- (228.5580,123.9170) -- (223.2250,123.9170) -- (223.2250,123.2770) -- cycle(233.8920,123.2770) -- (239.2250,123.2770) -- (239.2250,123.9170) -- (233.8920,123.9170) -- (233.8920,123.2770) -- cycle(244.5580,123.2770) -- (249.8920,123.2770) -- (249.8920,123.9170) -- (244.5580,123.9170) -- (244.5580,123.2770) -- cycle(20.5582,123.2740) -- (25.8915,123.2740) -- (25.8915,123.9140) -- (20.5582,123.9140) -- (20.5582,123.2740) -- cycle;



\path[draw=black,line join=round,line width=0.512pt] (252.0080,313.1890) .. controls (252.0080,373.1340) and (203.4130,421.7300) .. (143.4670,421.7300) .. controls (83.5218,421.7300) and (34.9263,373.1340) .. (34.9263,313.1890);



\path[fill=foo,line join=round,line width=0.256pt] (34.5615,317.0200) -- (34.5614,314.3750) -- (35.2014,314.3750) -- (35.2015,317.0200) -- (34.5615,317.0200) -- cycle(34.5616,327.6870) -- (34.5615,322.3540) -- (35.2015,322.3540) -- (35.2016,327.6870) -- (34.5616,327.6870) -- cycle;



\path[draw=black,line join=round,line width=0.512pt] (49.5931,125.0660) .. controls (49.5931,69.1502) and (94.9222,23.8211) .. (150.8380,23.8211) .. controls (206.7540,23.8211) and (252.0830,69.1502) .. (252.0830,125.0660);



\path[fill=foo,line join=round,line width=0.160pt] (49.5098,122.4650) .. controls (50.0234,122.4650) and (50.4398,122.8810) .. (50.4398,123.3950) .. controls (50.4398,123.9090) and (50.0234,124.3250) .. (49.5098,124.3250) .. controls (48.9962,124.3250) and (48.5798,123.9090) .. (48.5798,123.3950) .. controls (48.5798,122.8810) and (48.9962,122.4650) .. (49.5098,122.4650) -- cycle;



\path[fill=foo,line join=round,line width=0.160pt] (34.9047,312.7700) .. controls (35.4184,312.7700) and (35.8348,313.1860) .. (35.8348,313.7000) .. controls (35.8348,314.2140) and (35.4184,314.6300) .. (34.9047,314.6300) .. controls (34.3911,314.6300) and (33.9748,314.2140) .. (33.9748,313.7000) .. controls (33.9748,313.1860) and (34.3911,312.7700) .. (34.9047,312.7700) -- cycle;



\path[cm={{1.0,0.0,0.0,1.0,(74.0,421.0)}}] (0.0000,0.0000) node[above right] () {$\varphi_2$};



\path[cm={{1.0,0.0,0.0,1.0,(241.0,14.0)}}] (0.0000,0.0000) node[above right] () {$\varphi_3$};



\path[fill=foo,line join=round,line width=0.256pt] (251.7540,317.0240) -- (251.7520,311.6910) -- (252.3920,311.6910) -- (252.3940,317.0240) -- (251.7540,317.0240) -- cycle(251.7490,306.3580) -- (251.7460,301.0240) -- (252.3860,301.0240) -- (252.3890,306.3570) -- (251.7490,306.3580) -- cycle(251.7440,295.6910) -- (251.7410,290.3580) -- (252.3810,290.3570) -- (252.3840,295.6910) -- (251.7440,295.6910) -- cycle(251.7390,285.0240) -- (251.7360,279.6910) -- (252.3760,279.6910) -- (252.3790,285.0240) -- (251.7390,285.0240) -- cycle(251.7330,274.3580) -- (251.7310,269.0240) -- (252.3710,269.0240) -- (252.3730,274.3570) -- (251.7330,274.3580) -- cycle(251.7280,263.6910) -- (251.7250,258.3580) -- (252.3650,258.3570) -- (252.3680,263.6910) -- (251.7280,263.6910) -- cycle(251.7230,253.0240) -- (251.7200,247.6910) -- (252.3600,247.6910) -- (252.3630,253.0240) -- (251.7230,253.0240) -- cycle(251.7180,242.3580) -- (251.7150,237.0240) -- (252.3550,237.0240) -- (252.3580,242.3570) -- (251.7180,242.3580) -- cycle(251.7120,231.6910) -- (251.7100,226.3580) -- (252.3500,226.3570) -- (252.3520,231.6910) -- (251.7120,231.6910) -- cycle(251.7070,221.0240) -- (251.7050,215.6910) -- (252.3450,215.6910) -- (252.3470,221.0240) -- (251.7070,221.0240) -- cycle(251.7020,210.3580) -- (251.6990,205.0240) -- (252.3390,205.0240) -- (252.3420,210.3570) -- (251.7020,210.3580) -- cycle(251.6970,199.6910) -- (251.6940,194.3580) -- (252.3340,194.3570) -- (252.3370,199.6910) -- (251.6970,199.6910) -- cycle(251.6920,189.0240) -- (251.6890,183.6910) -- (252.3290,183.6910) -- (252.3320,189.0240) -- (251.6920,189.0240) -- cycle(251.6860,178.3580) -- (251.6840,173.0240) -- (252.3240,173.0240) -- (252.3260,178.3570) -- (251.6860,178.3580) -- cycle(251.6810,167.6910) -- (251.6790,162.3580) -- (252.3190,162.3570) -- (252.3210,167.6910) -- (251.6810,167.6910) -- cycle(251.6760,157.0240) -- (251.6730,151.6910) -- (252.3130,151.6910) -- (252.3160,157.0240) -- (251.6760,157.0240) -- cycle(251.6710,146.3580) -- (251.6680,141.0240) -- (252.3080,141.0240) -- (252.3110,146.3570) -- (251.6710,146.3580) -- cycle(251.6650,135.6910) -- (251.6630,130.3580) -- (252.3030,130.3570) -- (252.3050,135.6910) -- (251.6650,135.6910) -- cycle(251.6600,125.0240) -- (251.6600,123.4650) -- (252.2990,123.4650) -- (252.3000,125.0240) -- (251.6600,125.0240) -- cycle(251.7590,327.6910) -- (251.7570,322.3580) -- (252.3970,322.3570) -- (252.3990,327.6910) -- (251.7590,327.6910) -- cycle;



\path[draw=black,line join=round,line width=0.512pt] (252.0790,123.7970) -- (252.0790,313.6420);



\path[fill=foo,line join=round,line width=0.160pt] (252.0400,122.4420) .. controls (252.5530,122.4420) and (252.9700,122.8580) .. (252.9700,123.3720) .. controls (252.9700,123.8850) and (252.5530,124.3020) .. (252.0400,124.3020) .. controls (251.5260,124.3020) and (251.1100,123.8850) .. (251.1100,123.3720) .. controls (251.1100,122.8580) and (251.5260,122.4420) .. (252.0400,122.4420) -- cycle;



\path[fill=foo,line join=round,line width=0.160pt] (251.9350,312.7470) .. controls (252.4480,312.7470) and (252.8650,313.1630) .. (252.8650,313.6770) .. controls (252.8650,314.1900) and (252.4480,314.6070) .. (251.9350,314.6070) .. controls (251.4210,314.6070) and (251.0050,314.1900) .. (251.0050,313.6770) .. controls (251.0050,313.1630) and (251.4210,312.7470) .. (251.9350,312.7470) -- cycle;




\end{tikzpicture}


  \caption[Fixed-wing aerial robot plan to cover a regular polygon with four sides]{Fixed-wing aerial robot plan to cover a regular polygon with four sides. The plan covers the polygon with the same principle in \fref{fig:bm-like_pb}{Figure} but respecting turning constraints. The polygon is covered by iterating the primitive paths (gray lines) and triggering points (dashed blue line). The black line is the planned flight. The height and length of the polygon are $y_{\Gamma_3}-y_{\Gamma_1}$ and $lx_\mathbf{d}/4$.}
  \label{fig:zambo-like_pb}
\end{figure}

The coverage problem can be solved using the paths in \frefeq{eq:basic-plan}, the triggering points in \frefeq{eq:basic-plan-trigs}, the remaining paths $\varphi_5,\varphi_6,\dots,\varphi_l$, and triggering points $\mathbf{p}_{\Gamma_5},\mathbf{p}_{\Gamma_6},\dots,\mathbf{p}_{\Gamma_l}$ defined similarly to \frefeqM{eq:basic-plan}{eq:basic-plan-trigs}. A generic solution for the coverage problem defined as an iterated pattern contains the primitive paths
\begin{subequations}\label{eq:line-gene}\begin{align}
  \varphi_i(\mathbf{p}(t))&:=x-x_{\Gamma_1}-\lfloor i/4\rfloor x_\mathbf{d},\\
  \varphi_{i+2}(\mathbf{p}(t))&:=x-x_{\Gamma_2}-\lfloor i/4\rfloor x_\mathbf{d},
\end{align}
\end{subequations}
$\forall i\in\{1,5,9,\dots\}$. $x_\mathbf{d}$ is a shift on the $x$-axis, $\lfloor i/4\rfloor$ is the integer division. The expressions in \frefeq{eq:line-gene} correspond to the generalizations of the lines in \frefeq{eq:line1} and \frefeq{eq:line2}. The generalizations of the circles in \frefeq{eq:circ1} and \frefeq{eq:circ2} are
\begin{subequations}\label{eq:circ-gene}\begin{align}
  \varphi_{i+1}(\mathbf{p}(t))&:=(x-x_{\Gamma_1}-r_1-\lfloor i/4\rfloor x_\mathbf{d})^2+(y-y_{\Gamma_1}-\lfloor i/4\rfloor y_\mathbf{d})^2-r_1^2,\\
  \varphi_{i+3}(\mathbf{p}(t))&:=(x-x_{\Gamma_2}+r_2-\lfloor i/4\rfloor x_\mathbf{d})^2+(y-y_{\Gamma_3}-\lfloor i/4\rfloor y_\mathbf{d})^2-r_2^2,\label{eq:second-circ-gene}
\end{align}
\end{subequations}
where index $i$ is defined the same way as in \frefeq{eq:line-gene}, along with the shift (additionally, $y_\mathbf{d}$ is a shift on the $y$-axis) and integer division. $r_1>r_2>\underline{r}$ are given radiuses of the circles $\varphi_2$ and $\varphi_4$ in \fref{fig:zambo-like_pb}{Figure}, and $\underline{r}$ is the turning radius in \fref{pb:cov-pb}{Definition}.

The triggering points can be expressed similarly with the expressions
\begin{subequations}\label{eq:trigs-gene}\begin{align}
  \mathbf{p}_{\Gamma_i}&:=(x_{\Gamma_1}+\lfloor i/4\rfloor x_\mathbf{d},y_{\Gamma_1}+\lfloor i/4\rfloor y_\mathbf{d}),\\
  \mathbf{p}_{\Gamma_{i+1}}&:=(x_{\Gamma_1}+2r_1+\lfloor i/4\rfloor x_\mathbf{d},y_{\Gamma_1}+\lfloor i/4\rfloor y_\mathbf{d}),\\
  \mathbf{p}_{\Gamma_{i+2}}&:=(x_{\Gamma_1}+2r_1+\lfloor i/4\rfloor x_\mathbf{d},y_{\Gamma_3}+\lfloor i/4\rfloor y_\mathbf{d}),\\
  \mathbf{p}_{\Gamma_{i+3}}&:=(x_{\Gamma_1}+2r_1-2r_2+\lfloor i/4\rfloor x_\mathbf{d},y_{\Gamma_3}+\lfloor i/4\rfloor y_\mathbf{d})\label{eq:last-trig-gene}.
\end{align}
\end{subequations}

We note from \fref{fig:zambo-like_pb}{Figure} and \frefeqM{eq:line-gene}{eq:trigs-gene} that $y_{\Gamma_3}-y_{\Gamma_1}$ is the height of the polygon to be covered, $2(r_1-r_2)$ the distance between the covering lines (which is equal to the shift $x_\mathbf{d}$), and $lx_\mathbf{d}/4$ is the length of the polygon if the number of stages $l$ is known.

The plan in \frefeqM{eq:line-gene}{eq:trigs-gene} is static: no path parameter allows altering the plan. We can transform such a plan with the addition of a path parameter $c_{4,1}$ relative to the radius of the second circle $\varphi_4$ in \fref{fig:zambo-like_pb}{Figure}
\begin{equation}\label{eq:radius-dynamic}
  r_2(c_{4,1}):=\sqrt{r^2+c_{4,1}},
\end{equation}
where $\underline{r}<r<r_1$ is a given positive constant initial radius and $c_{4,1}\in(\underline{r}^2-r^2,0]$. We assume that the highest value is thus $\overline{c}_{4,1}=0$, the lowest is strictly higher than the difference between the turning and constant initial radiuses squared $\underline{c}_{4,1}>\underline{r}^2-r^2$ (to respect the turning radius).

We note from the expression in \frefeq{eq:radius-dynamic} that \frefeq{eq:last-trig-gene} and \frefeq{eq:second-circ-gene} depend on the parameter $c_{4,1}$ (they contain $r_2$, which depends on $c_{4,1}$). They can be thereby dynamically replanned, resulting in an alteration of the quality of the coverage. They indeed change the distance between the survey lines.
We can bound the alteration with 
\begin{equation}\label{eq:path-const-c}
  c_{i,1}\in[\underline{c}_{4,1},\overline{c}_{4,1}]=:\mathcal{C}_{4,1}\subseteq(\underline{r}^2-r^2,0],\,\forall i,
\end{equation} 
where we assume for ease of notation that we can change the parameter in advance at any stage (thus we use $\forall i$ in the equation). We will see in \fref{cp:res}{Chapter} that $\underline{c}_{4,1}$ is usually chosen from empirical observations to alter the flight time in case of, e.g., sudden battery drop, while still respecting the turning radius.

\begin{sidewaysfigure}
  \rotatesidewayslabel
  \centering
  \input{figures/zambo-repla_pb.tikz}
  \vspace*{30pt}
  \caption[Alteration of a path parameter of the fixed-wing aerial robot's plan]{Alteration of a path parameter of the fixed-wing aerial robot's plan in \fref{fig:zambo-like_pb}{Figure}. The black line is the un-altered path up to the triggering point $\mathbf{p}_{\Gamma_3}$, where the path can be altered with the parameter $c_{4,1}$ relative to the radius $r_2$ of $\varphi_4$. The alteration changes the distance between the survey lines hence extending or shortening the flying time. The longest distance is then $2(r_1-r_2(\overline{c}_{4,1}))$, the shortest $2(r_1-r_2(\underline{c}_{4,1}))$. The same principle applies to path parameter $c_{8,1}$ for the next two survey lines, to $c_{12,1}$, and so on.}
  \label{fig:zambo-repla_pb}
\end{sidewaysfigure}

The concept is illustrated in \fref{fig:zambo-repla_pb}{Figure}. The black line is the un-altered path until the triggering point $\mathbf{p}_{\Gamma_3}$, where the path splits depending on the value of the path parameter $c_{4,1}$. The alteration of the plan shortens or extends the flying time and thus influences the energy consumption. Since the last path $\varphi_4$ is a function of the parameter $c_{4,1}$, the correct expression for \frefeq{eq:second-circ-gene} is 
\begin{equation}\label{eq:line-gene-param}\begin{split}
  \varphi_{i+3}(\mathbf{p}(t),c_{4,1}):=(&x-x_{\Gamma_2}+r_2(c_{4,1})-\lfloor i/4\rfloor x_\mathbf{d})^2+\\
  (&y-y_{\Gamma_3}-\lfloor i/4\rfloor y_\mathbf{d})^2-r_2(c_{4,1})^2.
\end{split}\end{equation}

The last triggering point $\mathbf{p}_{\Gamma_4}$ in \frefeq{eq:last-trig-gene} is likewise a function of the parameter $c_{4,1}$
\begin{equation}
  \mathbf{p}_{\Gamma_{i+3}}(c_{4,1}):=(x_{\Gamma_1}+2r_1-2r_2(c_{4,1})+\lfloor i/4\rfloor x_\mathbf{d},y_{\Gamma_3}+\lfloor i/4\rfloor y_\mathbf{d}),
\end{equation}
as it depends on the radius $r_2$ of the circle $\varphi_4$. The same applies to all the following functions $\varphi_5,\varphi_6,\varphi_7$ since these are all a function of the triggering point $\mathbf{p}_{\Gamma_4}$. The path $\varphi_8$ is then a function of the parameter $c_{4,1}$ and $c_{8,1}$ propagating the parameters for all the following paths $\varphi_9,\varphi_{10},\varphi_{11}$ and so on.
We discuss further the coverage with the Zamboni-like motion in \fref{sec:cov-path-plan}{Section}.

\subsection{Computations sub-plan}
\label{sec:computation-wise}

In the computations sub-plan, we define the computations that form the plan. In the precision agriculture use case, we are interested in monitoring the field in \fref{fig:plot2}{Figure}, detecting ground hazards, and communicating with other ground-based actors. We use a convolutional neural network (\Gls{acr:cnn})\findex{convolutional neural network} implemented in a ROS node\findex{ROS!node} to detect the ground hazards. The CNN detection uses image frames from a downward-facing camera mounted on the aerial robot. We assume that there is one centralized workstation on the ground to communicate with the ground-based actors and that the communication between the aerial robot and the centralized workstation\findex{workstation} occurs on another ROS node, which uses the technical standard for wireless communication IEEE 802.11\findex{IEEE 802.11}~\citep{crow1997ieee}. It sends the detected images either unencrypted or using public-key infrastructure\findex{public-key infrastructure} for encryption\findex{encryption}. We refer to these two computations as CNN and encryption ROS nodes here and discuss the implementation in \fref{cp:res}{Chapter}. The schedule for the CNN ROS node is parametrized by the FPS rate at which the frames are sent from the camera and for the encryption ROS node by a binary value that indicates whenever the encryption is enabled.

There are thus two computations parameters. A computation parameter is the FPS rate, and another computation parameter is the encryption binary value. The CNN ROS node's computation parameter is $c_{i,2}$, and the encryption ROS node's computation parameter is $c_{i,3}$.

The upper and lower bounds of $c_{i,3}$ are 
\begin{equation}\label{eq:encr-comp-const}
  c_{i,3}\in\mathcal{S}_{i,3}:=\{0,1\},\,\forall i,
\end{equation}
the parameter value thus indicates if the encryption is active (one) or data are sent unencrypted (zero).

For the upper and lower bound of $c_{i,2}$, we first note from the path plan in \fref{sec:path-wise}{Section} that during the turns the aerial robot is not surveying the polygon. We thus place different constraints for different paths. For the circles in \frefeq{eq:circ-plan} and \frefeq{eq:circ2-plan}, the computations parameters constraint sets are 
\begin{equation}
  c_{i+1,2},c_{i+3,2}\in\mathcal{S}_{i+1,2}=\mathcal{S}_{i+3,2}:=\{0\},\,\forall i\in\{1,5,9,\dots\},
\end{equation}
and thus the CNN ROS node is not searching for any hazard when it flies out of the polygon. On the contrary, for the lines in \frefeq{eq:line-plan} and \frefeq{eq:line2-plan} 
\begin{equation}\label{eq:cnn-comp-const}
c_{i,2},c_{i+2,2}\in\mathcal{S}_{i,2}=\mathcal{S}_{i+2,2}:=[2,10],\,\forall i\in\{1,5,9,\dots\},
\end{equation} 
the CNN ROS node is detecting hazards on the ground with an FPS rate between two and ten.

Energy-wise we expect the highest configuration of computations parameters to correspond to the highest instantaneous energy consumption. We use \powprof{}, the profiling and modeling tool that we briefly outlined in \fref{sec:outline}{Section} and introduce in \fref{sec:comp-ener-model}{Section}, to measure and model the future energy consumption of different computations' configurations.

\subsection{Coverage plan with paths and computations sub-plans}

The plan for the precision agriculture use case is composed of stages, containing the primitive paths in \frefeqM{eq:line-gene}{eq:circ-gene} and the parameter-dependent version in \frefeq{eq:line-gene-param}. It further contains the path parameter $c_{i,1},\,\forall i\in\{4,8,\dots\}$ and the computations parameters $c_{i,2}$ and $c_{i,3},\,\forall i\in\{1,2,\dots\}$. In particular, the stages corresponding to the primitive paths are 
\begin{subequations}\label{eq:ex-pb-stages}\begin{align}
  \Gamma_i&:=\{\varphi_i(\mathbf{p}(t),c_{4,1},c_{8,1},\dots,c_{i-1,1}),c_{i,2},c_{i,3}\},\label{eq:line-plan}\\
  \Gamma_{i+1}&:=\{\varphi_{i+1}(\mathbf{p}(t),c_{4,1},c_{8,1},\dots,c_{i-1,1}),c_{i+1,2},c_{i+1,3}\},\label{eq:circ-plan}\\
  \Gamma_{i+2}&:=\{\varphi_{i+2}(\mathbf{p}(t),c_{4,1},c_{8,1},\dots,c_{i-1,1}),c_{i+2,2},c_{i+2,3}\},\label{eq:line2-plan}\\
  \Gamma_{i+3}&:=\{\varphi_{i+3}(\mathbf{p}(t),c_{4,1},c_{8,1},\dots,c_{i-1,1},c_{i+3,1}),c_{i+3,2},c_{i+3,3}\},\label{eq:circ2-plan}
\end{align}
\end{subequations}
$\forall i\in\{1,5,9,\dots\}$. The path constraint set for the path parameter $c_{i,1}$ is then in \frefeq{eq:path-const-c}, the computation constraint set for the computation parameter $c_{i,2}$ in \frefeq{eq:cnn-comp-const}, and for the computation parameter $c_{i,3}$ in \frefeq{eq:encr-comp-const}. The triggering points
\begin{subequations}\label{eq:ex-pb-trigs}\begin{align}
  \mathbf{p}_{\Gamma_i}(c_{4,1},\dots,c_{i-1,1})&:=(x_{\Gamma_{i-4}}(c_{4,1},\dots,c_{i-1,1})+\lfloor i/4\rfloor x_\mathbf{d},y_{\Gamma_1}+\lfloor i/4\rfloor y_\mathbf{d}),\\
  \mathbf{p}_{\Gamma_{i+1}}(c_{4,1},\dots,c_{i-1,1})&:=(x_{\Gamma_{i-4}}(c_{4,1},\dots,c_{i-1,1})+2r_1+\lfloor i/4\rfloor x_\mathbf{d},y_{\Gamma_1}+\lfloor i/4\rfloor y_\mathbf{d}),\hspace*{-1ex}\\
  \mathbf{p}_{\Gamma_{i+2}}(c_{4,1},\dots,c_{i-1,1})&:=(x_{\Gamma_{i-4}}(c_{4,1},\dots,c_{i-1,1})+2r_1+\lfloor i/4\rfloor x_\mathbf{d},y_{\Gamma_3}+\lfloor i/4\rfloor y_\mathbf{d}),\\
  \begin{split}
  \mathbf{p}_{\Gamma_{i+3}}(c_{4,1},\dots,c_{i+3,1})&:=(x_{\Gamma_{i-4}}(c_{4,1},\dots,c_{i-1,1})+2r_1-2r_2(c_{i+3,1})+\lfloor i/4\rfloor x_\mathbf{d},\\
  &\hspace*{4ex}y_{\Gamma_3}+\lfloor i/4\rfloor y_\mathbf{d}).\end{split}
\end{align}
\end{subequations}
$\forall i\in\{5,9,\dots\}$. The initial points $x_{\Gamma_1},y_{\Gamma_1}$ and $y_{\Gamma_3}$ are given along the shift $\mathbf{d}=(x_\mathbf{d},y_\mathbf{d})$, the radius of the first circle $r_1$, and the last triggering point $\mathbf{p}_l$. The function $r_2$, which returns the radius of the second circle and it is a function of the path parameter $c_{i,1}$, is in \frefeq{eq:radius-dynamic}. It contains $r$, which is likewise given (we can estimate $r$ from the desired distance of the covering lines $r=r_1-x_\mathbf{d}/2$).

The solutions to the planning problem are thus the three trajectories $c_i^*(t)=\{c_{i,1}^*(t),$ $c_{i,2}^*(t),$ $c_{i,3}^*(t)\},\,\forall i\in\{1,2,\dots\}$ that are energy-aware. 
Since each quadruple of stages in \frefeq{eq:ex-pb-stages} and triggering points in \frefeq{eq:ex-pb-trigs} depends only on the last path parameter (of the quadruple), we further assume that $c_{1,1}=c_{1,2}=c_{1,3}=c_{1,4}$ and more generally 
\begin{equation}\label{eq:very-last-assmp}
c_{i,1}=c_{i+1,1}=c_{i+2,1}=c_{i+3,1},\,\forall i\in\{1,5,9,\dots\}.
\end{equation}

\fref{cp:dyn}{Chapter} generalizes the plan in this section with coverage of an arbitrary polygon and derives the optimal configuration $c_i^*$. \fref{cp:res}{Chapter} revisits the assumption in \frefeq{eq:very-last-assmp} and the two sub-plans in an experimental implementation of the agricultural use case.


%%%%%%%%%%%%%%%%%
\section{Summary}

In this chapter, we defined the planning and coverage problems. The solution to the coverage problem is a cover tour; to the planning problem, the optimal configurations of the path and computations parameters. Both are interconnected and require some basic concepts, including plan and path functions. The plan is composed of stages, triggering, and final points and parameters for the replanning itself. At each stage, the robot flies a path function and computes a schedule. Some parameters parametrize the path, and some parameters parametrize the computations. Bounds limit these parameters due to physical impediments of the computing hardware and the aerial robot. Alternatively to defining all the stages manually, it is possible defining an autonomous use case with some periodicity, using primitive paths and a shift. We illustrated the concept with the precision agriculture use case. Here, we provided two sub-plans. The first contains the paths constructed with the Zamboni-like motion for fixed-wing aerial robots with strict turning radius constraints. The second contains the computations with the hazards detection and communication with other ground-based actors. There is one parameter to alter the quality of the coverage with the radius in the Zamboni-like motion and two parameters to alter the computations of the ground hazards detection and communication.

%---
%$$$

