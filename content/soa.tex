
%%%%%%%%%%%%%%%%%%%%%%
%                    %
% State of the Art   %
%                    %
\chapter{\color{red}State of the Art}
\label{cp:soa}

%\begin{highlight}
%    \begin{st}
%        Pasted energy modeling in \fref{cp:soa:ene-mod}{Section} from~\citep{seewald2019coarse}, rest dummy text.
%    \end{st} 
%\end{highlight}

\lettrine{A}{a}


\section{\color{cyan}Energy Modeling}
\label{cp:soa:ene-mod}

Marowka developed a power-metrics based analytical model, to exploit the possibility of considerably increasing energy efficiency by choosing an optimal chip configuration, which we extended to evaluate the impact of different architectural design choices on energy efficiency~\citep{marowka2017energy}. Marowka's theoretical contribution shows three processing schemes for heterogeneous computing with a comparison of their respective energy efficiency. Variations in chip configurations are done to investigate the impact on energy, power, and performance. Symmetric processor scheme consists of only a multicore CPU. Asymmetric CPU-GPU processor scheme consists of both CPU and GPU on the hardware side, and of a program running on CPU or GPU, but not on both in the same time interval, on the software side. CPU-GPU simultaneous processing scheme consists of a program running on CPU and GPU simultaneously. Our work extends and starts from Marowka's approach by building an experimental method to the CPU-GPU simultaneous processing scheme.

For evaluating the effects of a battery as an energy source, we used the work done by~\citep{rao2003battery}. Their work summarizes state-of-the-art battery modeling into four classes of models that capture the battery state and its non-linearities. The lowest class contains the physical models that are accurate and model battery state evolution through a set of ordinary and partial differential equations. However, they suffer from a significant level of complexity that reflects on the time needed to produce predictions. The work proceeds by showing empirical models, that predict battery state from empirical trials. The third class consists of abstract models that we incorporated into our approach, in particular, by deriving the equation from the model developed by~\citep{hasan2018exogenous} (they model battery state through an equivalent electrical circuit and its evolution in time). The fourth class consists of mixed models where experimental data are collected and subsequently refined with analytical expressions to determine the models' parameters.

System-level optimization techniques, such as dynamic voltage scaling, have been used for lowering the power consumption~\citep{hong1999power, luo2001battery, chowdhury2005static}. These techniques are available for hardware featuring dynamic voltage scaling and aim to achieve higher energy efficiency by including information about configuration parameters into the scheduler. They however focus on homogeneous systems, unlike our work which is designed to work for heterogeneous systems. This approach to modeling has nevertheless been extended to include GPU features~\citep{hong2010integrated}, to heterogeneous systems~\citep{bailey2014adaptive}, to optimal software partitioning~\citep{goraczko2008energy}, and by Wu et al.\ to machine learning techniques~\citep{wu2015gpgpu}. The work by Wu et al.\ mostly relies on neural networks and has been introduced only recently in the field of power estimation and modeling for heterogeneous systems. In particular, for a collection of applications, Wu et al.\ train a neural network by measuring a number of performance counters for different configurations. Even if these techniques perform well for defining static optimization strategies, they are generally not suitable for heterogeneous parallel systems in aerial robotics. In these cases, systems suffer from a considerable level of uncertainty, for which reason a statically defined energy model often would not model the real energy behavior. An overview of energy estimation in the context of machine learning approaches has recently been presented by~\citep{garcia2019estimation}. They present a literature review motivated by a belief that the machine learning community is unfamiliar with energy models, but do not relate to GPU-featured devices nor in general heterogeneous devices. In contrast, in our work we aim at automating the generation of application-level power estimation models that can be adapted for machine learning algorithms. Our approach does not yet take detailed scheduling decisions into account, unlike the black-box approach for CPU-GPU energy-aware scheduling presented by~\citep{barik2016black}: they model the power by relating execution time to power consumption but otherwise do not focus on energy models.

Our approach shares the same principle of differentiating the microcontroller from the companion computer with~\citep{mei2004energy, mei2005case}. The controller acts on the actuators and reads the sensors, while the companion computer (can be found with different names in literature, such as secondary or embedded computer), performs computationally heavy operations. A similar approach for mobile robots is presented by~\citep{dressler2005energy}. However, both contributions neither elaborate further on computational elements of a heterogeneous platform, such as GPU, nor focus on different robots except the one under analysis.

The majority of other contributions in the literature focus on optimizing motion planning to increase power efficiency. For instance, approaches to minimize UAV power consumption, such as the work by~\citep{kreciglowa2017energy}, aim to determine the best trajectory generation method for an aerial vehicle to travel from one configuration to another. Uragun suggests the use of power-efficient components~\citep{uragun2011energy}: an energy-efficient UAV system can either be built using conceptual product development with emerging technologies or using energy-efficient components. Kanellakis et al. affirm that integrating visual sensors in the UAV ecosystem still lacks solid experimental evaluation~\citep{kanellakis2017survey}. They suggest that to save energy, the available payload for sensing and computing has to be restricted. Our approach towards energy modeling shares a similar principle as the one presented by~\citep{sadrpour2013mission, sadrpour2013experimental} for Unmanned Ground Vehicles or UGVs. They propose a linear regression-based technique in the absence of real measurements and a Bayesian networks-based one in their presence. We used a simplified approximation technique to limit the number of computations needed while focusing rather on an accurate battery prediction.

To validate our approach and quantify its outcomes, we used the models previously developed for fine-grained energy modeling by~\citep{nunez2013enabling}, and~\citep{nikov2015evaluation} respectively. In summary, fine-grained energy modeling uses hardware event registers to capture the CPU state under a representative workload. The energy-modeling consists of three stages. In data collection, the first stage, a benchmark runs on the platform and data are collected. The second and third stage, data processing and model generation, are performed offline on a different architecture. In these two stages, data are analyzed and a model that predicts possible future usage is generated. 

Calore et al. develop an approach for measuring power efficiency for High-Performance Computing or HPC systems~\citep{calore2015energy}. An external board is used to measure the power consumption, while the data are collected from NVIDIA Jetson TK1 board running one benchmark. Our initial analysis presented at HLPGPU 2019 was made using a similar technique~\citep{seewald2019hlpgpu}. A shunt resistor and digital multimeter integrated into the external board was used to evaluate the power efficiency. In this paper we extend our experiments to use internal power monitors and address a broader range of platforms. We now build a proper energy model that reflects the computational behavior of the device under study and shows the energy evolution. An early report on our work has been presented in the TeamPlay project's deliverable D4.3~\citep{teamplay} ``Report on Energy, Timing and Security Modeling of Complex Architectures''.


\section{\color{cyan}Motion Planning}

Planning algorithms literature for mobile robots includes topics such as trajectory generation and path planning. Generally, the algorithms select an energy-optimized trajectory~\cite{mei2004energy}, e.g., by maximizing the operational time~\cite{wahab2015energy}. However, they apply to a small number of robots~\cite{kim2005energy} and focus exclusively on planning the trajectory~\cite{kim2008minimum}, despite compelling evidence for the energy consumption also being significantly influenced by computations~\cite{mei2005case}. Given the availability of powerful GPU-equipped mobile hardware~\cite{rizvi2017general}, the use of computations is expected to increase in the near future~\cite{abramov2012real,satria2016real,jaramillo2019visual}. More complex planning, which includes a broader concept of the plan being a set of tasks and a path, all focus on the trajectory~\cite{mei2005case,mei2006deployment} and apply to a small number of robots~\cite{sadrpour2013mission,sadrpour2013experimental}. For UAVs specifically, rotorcrafts have also gained interest in terms of algorithms for energy-optimized trajectory generation~\cite{morbidi2016minimum,kreciglowa2017energy}. 


\section{\color{red}Planning with Dynamics}


\section{\color{red}Planning for Autonomous Aerial Robots}

\subsection{\color{red}Flight controllers}

\subsection{\color{orange}Energy models in aerial robotics}

\subsection{\color{orange}State estimation in aerial robotics}

\subsection{\color{orange}Optimal control in aerial robotics}


\section{\color{red}Planning Computations with Motion}


\section{\color{red}Summary}

