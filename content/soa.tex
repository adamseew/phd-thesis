
%%%%%%%%%%%%%%%%%%%%%%
%                    %
% State of the Art   %
%                    %
\chapter{State of the Art}
\label{cp:soa}

In this chapter, we discuss the state of the art in dynamic energy planning for autonomous aerial robots. Generally, planning algorithms in robotics center around robot motion planning and deal with problems such as swarms, dynamics, and uncertainty~\citep{lavalle2006planning}. Although there are several contributions applied to a variety of mobile robots, we primarily focus on the literature with the objective of planning dynamically the path along with the computations. We especially emphasize approaches applied to aerial robots with energy constraints operating in coverage problems. To this end, we split the chapter into multiple sections and replicate our workflow throughout the topic. Initially, we analyzed some contributions that quantify the energy consumption of computing hardware carried by mobile robots. Modeling the energy of these devices has laid the foundation for our dynamic energy planning. In \fref{sec:soa-ene-mod}{Section}, we report our findings. We then briefly discuss some approaches for motion planning and planning with dynamical systems in \fref{sec:soa-motion-pl}{Sections}\fref{sec:soa-dynamics-pl}{--\hspace{-.8ex}}. \fref{sec:soa-aerial-pl}{Section} is entirely dedicated to planning for aerial robots. In particular, we outline our findings on how planning in the literature occurs in terms of defining an autonomous flight plan. We then analyze the literature on energy modeling, state estimation, and optimal control for aerial robots. Although simultaneous dynamic planning of computations and motion is underrepresented, some recent contributions have proposed various techniques and motivated our analysis. We report these in detail in \fref{sec:soa-comp-motion-pl}{Section}.

This chapter connects to the remainder of this work as follows. In \fref{sec:soa-ene-mod}{Section} we discuss the state of the art on topics that allowed us to derive a dynamic energy planning approach for autonomous aerial robots. Based on these findings, we later propose a computations energy modeling technique to derive future energy consumption of mobile computing elements along with motion in \fref{cp:model}{Chapter}. Motion planning in the context of mobile robots and planning under dynamics are the basis for the derivation of the optimal configuration in \fref{cp:opt}{Chapter} of the plan and the planning problem that we defined in \fref{cp:pb}{Chapter}.       


\section{Computations Energy Modeling}
\label{sec:soa-ene-mod}

There are several different energy modeling and optimization approaches for computations, usually under the topic of energy efficiency for computing hardware. 
Such hardware is carried by aerial robots that we analyze in this work and is often composed of heterogeneous elements: one or more CPUs and a GPU as we outlined in \fref{sec:motivation}{Section}. Energy efficiency is critical for both battery constrained devices~\citep{seewald202Xenergy} and a limiting factor in improving further computing performance~\citep{horowitz2014computing}. We split some of the available approaches in the literature into different classes, depending on their modeling and optimization approach. Due to the unpredictable nature of the heterogeneous elements, many contributions to energy modeling observe hardware characteristics and perform physical energy consumption measurements to derive an energy model~\citep{teamplay}. We analyze some of these contributions first. They treat the heterogeneous elements altogether and are of particular interest for our approach that likewise shares the same principle. We then analyze the techniques that focus on the energy of GPUs. Finally, we analyze techniques that treat CPUs. They are based on dynamic voltage scaling (\Gls{acr:dvs})\findex{dynamic voltage scaling}, a technique that scales down both the supply voltage and the frequency when there is no high computations demand~\citep{flautner2001automatic, chen2009fundamentals}.

\subsection{Heterogeneous elements modeling}

Modern computing hardware energy modeling and optimization techniques often deal with the heterogeneous computing elements altogether using statistical tools~\citep{teamplay}. Such tools are inexpensive to deploy and relatively accurate in predicting future computations energy consumption~\citep{seewald2019coarse}. Although there are further optimizations available by looking at the elements (CPUs, GPUs) separately, these are often application and hardware-dependent. Instead, we focus on a generic computations energy model that can be used independently of the hardware and computations under analysis. 

A model for heterogeneous elements is derived in~\citep{marowka2017energy}. The model is based on some power metrics: the scaled speedup, scaled performance per watt, and scaled performance per joule. The author increases energy efficiency by choosing the configuration of the heterogeneous components: for instance, by enabling computations only on CPU cores, and hence, investigates the impact of different architectural choices on energy efficiency. In particular, the model is used to analyze the energy of three processing schemes: symmetric, asymmetric, and simultaneous. The former processing scheme utilizes merely a multicore CPU. Asymmetric processing scheme both CPU and GPU. The software benchmark on this scheme consists of running a program on one of the computing elements at a time. The latter processing scheme is similar to the previous one but the software benchmark runs on CPU and GPU simultaneously. Our work extends the model and builds an experimental method to the simultaneous processing scheme.

A more sophisticated statistical model that relies on multivariate linear regression for heterogeneous elements is derived in~\citep{bailey2014adaptive}. The model aims to allow selecting the application's configuration that maximizes performance under given power constraints. It is trained offline with a small set of benchmarks and works across multiple devices. The overall flow of the proposed approach is composed of two stages. The first stage is the offline training stage. It utilizes a small training set of benchmarks split into clusters. The model is built then with a regression per each cluster. After this stage follows the online predicting stage. The latter utilizes the model to predict the power and performance of a large set of applications (opposed to the relatively small number of benchmarks in the offline stage). Our work similarly models a subset of samples and infers properties of the entire search space. However, it focuses on heterogeneous mobile devices that we use for aerial robots.

A resource model for heterogeneous mobile devices developed in~\citep{goraczko2008energy} considers both the time and power of a given run-time mode (the authors use the term mode rather than configuration in~\citep{marowka2017energy,bailey2014adaptive} and in our work). Energy-wise, the resource model utilizes dynamic voltage scaling (DVS) as the other models in \fref{sec:soa-cpu}{Section} but accounts for heterogeneous systems. The approach models multiple processors with a state machine then used to derive a software partitioning problem--the problem of deriving the optimal mode. The problem is solved with an optimization technique: integer linear programming (ILP)\findex{integer linear programming}. The optimization occurs over an energy cost and with given deadline constraints. The intuition of formally defining the problem of deriving the optimal mode is similar in our work, which we further merge with the path planning for our dynamic energy planning.

Some computations-specific modeling approaches have been developed in the context of machine learning in~\citep{yang2017method}, and more recently surveyed in~\citep{garcia2019estimation}. Indeed, recent approaches are emerging to model the energy consumption of machine learning computations. The latter is a literature review motivated by the lack of appropriate tools to build and measure power models in existing machine learning suites. It describes the state of the art of energy estimation for convolutional neural networks (\Gls{acr:cnn}s) and data mining\findex{data mining}. In~\citep{yang2017method}, the approach evaluates an energy model of deep neural networks (\Gls{acr:dnn}s)\findex{deep neural network} based on a network bitwidth, sparsity, and architecture. The methodology applies exclusively to DNNs, but has been extended and used with other CNNs in~\citep{yang2017designing} in an optimization loop to reduce the computations energy consumption.

A holistic approach for heterogeneous elements modeling is proposed in~\citep{ma2012holistic}. It achieves energy efficiency by splitting and distributing the workload among the heterogeneous elements, frequency scaling for the CPU and GPU, and \Gls{acr:dvs} for the CPU. GPU-side, the frequency is determined with a lightweight machine learning algorithm. Energy modeling in this approach is achieved by empirical means, using two power meters. The testbed under analysis--NVIDIA GeForce GPUs and AMD Phenom II CPUs--does not include built-in measurement units, and overall, the approach does not consider mobile computing heterogeneous hardware. Nevertheless, it is of interest in deriving insights on energy implications of energy efficiency of heterogeneous elements. 


\subsection{GPU features modeling}

An energy model that analyzes GPU features is derived in~\citep{hong2010integrated}. The contribution consists of an integrated power and performance prediction model to derive the optimal number of active processors for a given application. The model predicts performance per watt and the optimal number of cores to achieve energy savings~\citep{hong2010integrated}. The model does not account for mobile computing hardware.

A similar focus on GPU features is proposed in~\citep{wu2015gpgpu}.  The model utilizes machine learning techniques to evaluate the performance and estimate the power from measurements of real GPU hardware. In particular, the approach trains a neural network by measuring different performance counters for various GPU configurations of a collection of applications. The data gathered from one hardware are used to estimate the power and performance for multiple other GPU hardware. Like the approaches in~\fref{sec:soa-hete}{Section}, the approach is computation independent and requires no source code analysis. It performs well for defining static offline modeling strategies, yet for aerial robots with systems suffering uncertainty, a dynamic approach is often favorable~\citep{seewald2019coarse}. Moreover, machine learning is an energy-expensive computation itself~\citep{garcia2019estimation,yang2017method}, thus deterring similar approaches in our planning where we aim to preserve the energy as far as possible.

An empirical approach for energy evaluations of GPUs during various computations in CUDA\findex{CUDA} environment is proposed in~\citep{collange2009power}. The approach measures and analyzes how computations impact instantaneous energy consumption. It observes a significant energy impact of memory accesses and generally analyzes the energy cost of parallel GPU computations. It does not use the measured data to calibrate a model for energy estimation, nor focus on mobile computing devices.

Opposed to the previous approach of pure empirical measurements, \citep{luo2011performance} proposes an analytical model for energy and performance estimation of GPUs. The model contains execution time and energy consumption prediction sub-models, where the latter follows from the former. The final analytical model for the energy estimation multiplies the execution time and the estimated power derived using an analytical expression. The accuracy analysis, redefined by the authors with another analytical expression, shows a strong correlation between observation and the analytical model. We also derive an analytical expression. We further motivate such expression with empirical observation of energy data and employ a multivariate linear regression. We then focus on heterogeneous elements and mobile computing hardware.

---


\subsection{CPU features modeling}
\label{sec:soa-cpu}

Numerous other contributions investigate how to lower the power~\citep{hong1999power, luo2001battery, chowdhury2005static} by some system-level modeling and optimization techniques such as dynamic voltage scaling. They usually include into the scheduler information about configuration parameters~\citep{seewald2019coarse}. They mostly focus on homogeneous systems opposite to heterogeneous systems that we utilize in our work. 






 



For evaluating the effects of a battery as an energy source, we used the work done by~\citep{rao2003battery}. Their work summarizes state-of-the-art battery modeling into four classes of models that capture the battery state and its non-linearities. The lowest class contains the physical models that are accurate and model battery state evolution through a set of ordinary and partial differential equations. However, they suffer from a significant level of complexity that reflects on the time needed to produce predictions. The work proceeds by showing empirical models, that predict battery state from empirical trials. The third class consists of abstract models that we incorporated into our approach, in particular, by deriving the equation from the model developed by~\citep{hasan2018exogenous} (they model battery state through an equivalent electrical circuit and its evolution in time). The fourth class consists of mixed models where experimental data are collected and subsequently refined with analytical expressions to determine the models' parameters.


Our approach shares the same principle of differentiating the microcontroller from the companion computer with~\citep{mei2004energy, mei2005case}. The controller acts on the actuators and reads the sensors, while the companion computer (can be found with different names in literature, such as secondary or embedded computer), performs computationally heavy operations. A similar approach for mobile robots is presented by~\citep{dressler2005energy}. However, both contributions neither elaborate further on computational elements of a heterogeneous platform, such as GPU, nor focus on different robots except the one under analysis.

The majority of other contributions in the literature focus on optimizing motion planning to increase power efficiency. For instance, approaches to minimize UAV power consumption, such as the work by~\citep{kreciglowa2017energy}, aim to determine the best trajectory generation method for an aerial vehicle to travel from one configuration to another. Uragun suggests the use of power-efficient components~\citep{uragun2011energy}: an energy-efficient UAV system can either be built using conceptual product development with emerging technologies or using energy-efficient components. Kanellakis et al. affirm that integrating visual sensors in the UAV ecosystem still lacks solid experimental evaluation~\citep{kanellakis2017survey}. They suggest that to save energy, the available payload for sensing and computing has to be restricted. Our approach towards energy modeling shares a similar principle as the one presented by~\citep{sadrpour2013mission, sadrpour2013experimental} for Unmanned Ground Vehicles or UGVs. They propose a linear regression-based technique in the absence of real measurements and a Bayesian networks-based one in their presence. We used a simplified approximation technique to limit the number of computations needed while focusing rather on an accurate battery prediction.

To validate our approach and quantify its outcomes, we used the models previously developed for fine-grained energy modeling by~\citep{nunez2013enabling}, and~\citep{nikov2015evaluation} respectively. In summary, fine-grained energy modeling uses hardware event registers to capture the CPU state under a representative workload. The energy-modeling consists of three stages. In data collection, the first stage, a benchmark runs on the platform and data are collected. The second and third stage, data processing and model generation, are performed offline on a different architecture. In these two stages, data are analyzed and a model that predicts possible future usage is generated. 

Calore et al. develop an approach for measuring power efficiency for High-Performance Computing or HPC systems~\citep{calore2015energy}. An external board is used to measure the power consumption, while the data are collected from NVIDIA Jetson TK1 board running one benchmark. Our initial analysis presented at HLPGPU 2019 was made using a similar technique~\citep{seewald2019hlpgpu}. A shunt resistor and digital multimeter integrated into the external board was used to evaluate the power efficiency. In this paper we extend our experiments to use internal power monitors and address a broader range of platforms. We now build a proper energy model that reflects the computational behavior of the device under study and shows the energy evolution. An early report on our work has been presented in the TeamPlay project's deliverable D4.3~\citep{teamplay} ``Report on Energy, Timing and Security Modeling of Complex Architectures''.


\section{\color{cyan}Motion Planning}
\label{sec:soa-motion-pl}

Planning algorithms literature for mobile robots includes topics such as trajectory generation and path planning. Generally, the algorithms select an energy-optimized trajectory~\cite{mei2004energy}, e.g., by maximizing the operational time~\cite{wahab2015energy}. However, they apply to a small number of robots~\cite{kim2005energy} and focus exclusively on planning the trajectory~\cite{kim2008minimum}, despite compelling evidence for the energy consumption also being significantly influenced by computations~\cite{mei2005case}. Given the availability of powerful GPU-equipped mobile hardware~\cite{rizvi2017general}, the use of computations is expected to increase in the near future~\cite{abramov2012real,satria2016real,jaramillo2019visual}. More complex planning, which includes a broader concept of the plan being a set of tasks and a path, all focus on the trajectory~\cite{mei2005case,mei2006deployment} and apply to a small number of robots~\cite{sadrpour2013mission,sadrpour2013experimental}. For UAVs specifically, rotorcrafts have also gained interest in terms of algorithms for energy-optimized trajectory generation~\cite{morbidi2016minimum,kreciglowa2017energy}. 


\section{\color{red}Planning with Dynamics}
\label{sec:soa-dynamics-pl}

\section{\color{red}Planning for Autonomous Aerial Robots}
\label{sec:soa-aerial-pl}

\subsection{\color{red}Flight controllers}

\subsection{\color{orange}Energy models in aerial robotics}

\subsection{\color{orange}State estimation in aerial robotics}

\subsection{\color{orange}Optimal control in aerial robotics}


\section{\color{red}Planning Computations with Motion}
\label{sec:soa-comp-motion-pl}

\section{\color{red}Summary}

