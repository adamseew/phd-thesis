
%
%%%%%%%%%%%%%%
%            %
% Abstract   %
%            %
%%%%%%%%%%%%%%
%
\chapter*{Abstract}

This work aims to derive an energy-optimal path and a power-saving schedule for an aerial robot simultaneously, inflight, and under strict energy constraints. Although energy conservations techniques for mobile robots' motion planning and the scheduling for heterogeneous computing hardware carried by these robots have been studied, the close interaction between the two remains mostly unexplored. It is regarded in the available literature that there are classes of mobile robots where it could be advantageous to trade-off reduced resources for a still acceptable level of performance. 

Within mobile robots, aerial robots are particularly affected by various energy considerations. Generally, it would be required to land and recharge the battery in case of adverse energy-related events. Therefore, this work emphasizes aerial robots. It derives planning-scheduling energy awareness using optimal control techniques, regression analysis, and differential periodic energy modeling. The future computations energy prediction further derives an automatic profiling and modeling utility that generates overall energy, average power, and battery state of charge models in the function of a software configuration, allowing the integration in a data-flow computational network. The work is demonstrated on the problem of planning coverage in an autonomous precision agriculture use case, where a fixed-wing aerial robot flies over an agricultural field, detects hazards, and communicates the detections with other ground-based actors. The guidance on the coverage path relies on the theory of vector fields, and the overall approach is an algorithm that incorporates the battery, motion, and computations energy modeling along with gradient descent and optimal control. 

Planning-scheduling exhibits improved performance mitigating the effect of uncertainty in battery-powered aerial robots against the baseline of an aerial robot flying full coverage, requiring landing in the eventuality of sudden battery defect. Although specific, the approach can be generalized. The computations energy and battery models applied to different domains, whereas the differential periodic energy model, the guidance, and the overall planning-scheduling approach to a broad class of potential autonomous mobile robotics use cases.

%Abstract in a nutshell:
%What is the reason for writing the thesis?
%What are the current approaches and gaps in the literature?
%What are your research question(s) and aims?
%Which methodology have you used?
%What are the main findings?
%What are the main conclusions and implications?

\cleardoublepage   % tocpage on right-side

