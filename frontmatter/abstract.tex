
%
%%%%%%%%%%%%%%
%            %
% Abstract   %
%            %
%%%%%%%%%%%%%%
%
\chapter*{Abstract}

This work aims to derive an energy-optimal path and a power-saving schedule for an aerial robot simultaneously, inflight, and under strict energy constraints. Although energy conservations techniques for mobile robots' motion planning and the scheduling for heterogeneous computing hardware carried by these robots have been studied, the close interaction between the two remains mostly unexplored. It is regarded in the available literature that there are classes of mobile robots where it could be advantageous to trade-off reduced resources for a still acceptable level of performance. 

Within mobile robots, aerial robots are particularly affected by various energy considerations. Generally, it would be required to land and recharge the battery in case of adverse energy-related events. Therefore, this work emphasizes aerial robots. It derives planning-scheduling energy awareness using optimal control techniques, regression analysis, and differential periodic energy modeling. The future computations energy prediction further derives an automatic profiling and modeling utility that generates overall energy, average power, and battery state of charge models in the function of a software configuration, allowing the integration in a data-flow computational network. The work is demonstrated on the problem of planning coverage in an autonomous precision agriculture use case, where a fixed-wing aerial robot flies over an agricultural field, detects hazards, and communicates the detections with other ground-based actors. The guidance on the coverage path relies on the theory of vector fields, and the overall approach is an algorithm that incorporates the battery, motion, and computations energy modeling along with gradient descent and optimal control. 

Planning-scheduling exhibits improved performance mitigating the effect of uncertainty in battery-powered aerial robots against the baseline of flying full coverage, requiring landing in the eventuality of sudden battery defect. Although specific, the approach can be generalized. The computations energy and battery models applied to different domains, whereas the differential periodic energy model, the guidance, and the overall planning-scheduling approach to a broad class of potential autonomous mobile robotics use cases.

%Abstract in a nutshell:
%What is the reason for writing the thesis?
%What are the current approaches and gaps in the literature?
%What are your research question(s) and aims?
%Which methodology have you used?
%What are the main findings?
%What are the main conclusions and implications?

\cleardoublepage

% Abstract in Danish (credit: Ulrik)

\chapter*{Resum{\'e}}

\vspace*{-1.8ex}

Arbejdet i denne afhandling sigter mod at udlede kombinationen af en energioptimal rute og en str{\o}mbesparende eksekveringsplan for en aerorobot (drone), hvis flyvning skal foretages indenfor et fast energibudget. Der findes allerede teknikker til bev{\ae}gelsesplanl{\ae}gning for mobile robotter samt teknikker til planl{\ae}gningen af beregninger udf{\o}rt p{\aa} computerhardware b{\aa}ret af s{\aa}danne robotter, men interaktionen mellem disse to aspekter er stort set uudforsket. Det ses dog ud fra den tilg{\ae}ngelige litteratur, at der findes bestemte typer af mobile robotter, for hvilke det ville v{\ae}re fordelagtigt at kunne foretage en afvejning mellem ressourceforbruget ved henholdsvis bev{\ae}gelse og beregning, i relation til kvaliteten af det arbejde som robotten udf{\o}rer.
 
Aerorobotter er i s{\ae}rligt grad p{\aa}virket af behovet for energibesparelse. Det kan v{\ae}re %Ofte er det %endda 
%v{\ae}re 
n{\o}dvendigt at lande og genoplade batteriet i tilf{\ae}lde af et uforudset h{\o}jt str{\o}mforbrug. Denne afhandling fokuserer p{\aa} aerorobotter. Der udledes planl{\ae}gningsteknikker til energif{\o}lsom styring af aerorobottens bev{\ae}gelser og beregninger %ved hj{\ae}lp af 
vha. optimal kontrol, regressionsanalyse og differentiel periodisk energimodellering. Forudsigelse af beregningernes fremtidige energiforbrug foretages via et automatisk profilerings- og modelleringsv{\ae}rkt{\o}j, som modellerer det overordnede energiforbrug, gennemsnitseffekt samt batteriafladningen som funktion af robottens softwarenkonfiguration. Arbejdet demonstreres ved planl{\ae}gningen af en rute, som d{\ae}kker et geografisk omr{\aa}de, og er tilt{\ae}nkt anvendelse indenfor pr{\ae}cisionslandbrug. En fastvinget aerorobot flyver over en landbrugsmark, g{\o}r opm{\ae}rksom p{\aa} farer, og kommunikerer disse til andre jordbaserede akt{\o}rer. Aerorobottens d{\ae}kning af omr{\aa}det beskrives via vektorfelter, og den overordnede tilgang til problemets l{\o}sning udg{\o}res af en algoritme, der inkorporerer batteri-, bev{\ae}gelses- og beregningsenergimodellering sammen med gradientnedstigning og optimal kontrol.
 
Den kombinerede tilgang til planl{\ae}gning af rute og beregninger viser forbedret ydeevne, der hj{\ae}lper til at afb{\o}de effekten af usikkerhed i missioner udf{\o}rt med batteridrevne aerorobotter. Dette ses i forhold til den traditionelle l{\o}sning med en aerorobot, der ikke kan omplanl{\ae}gge, og derfor foretager fuld d{\ae}kning af omr{\aa}det, hvilket kan resultere i en n{\o}dlanding i tilf{\ae}lde af uforudsete h{\ae}ndelser s{\aa}som et delvist defekt batteri. Selvom tilgangen er specifik til aerorobotter kan den generaliseres. Energimodellerne for beregninger og batteriforbrug kan anvendes p{\aa} forskellige dom{\ae}ner. Den differentielle periodiske energimodel og den overordnede tilgang til planl{\ae}gning kan benyttes p{\aa} tv{\ae}rs af mange forskellige anvendelsesomr{\aa}der indenfor autonome mobile robotter.


\cleardoublepage   % tocpage on right-side

